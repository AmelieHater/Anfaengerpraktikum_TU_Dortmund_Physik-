%Abweichung Nr. 1:
%Abweichung Chi Nr.1:  (3.17+/-0.11)e+03
%Abweichung Chi Nr.1:  (7.6+/-0.4)e+03
%Abweichung Chi Nr.1:  2812+/-12
%
%Abweichung Nr.2:  311+/-21
%Abweichung Nr.2:  (1.5+/-0.5)e+02
%Abweichung Nr.2:  347+/-9

\section{Diskussion}
\label{sec:Diskussion}
Die Abweichungen werden mithilfe der Formel
\begin{equation*}
    \text{rel. Abweichung} = \frac{|\text{exp. Wert} - \text{theo. Wert}|}{\text{theo. Wert}}
  \end{equation*}
berechnet. Der eingestellte Wert der Güte $Q$ des Selektivverstärkers ist $100$, der durch die Messwerte ermittelte Wert ist $73,52$. Dies entspricht einer Abweichung
von $26,48 \,\%$.
