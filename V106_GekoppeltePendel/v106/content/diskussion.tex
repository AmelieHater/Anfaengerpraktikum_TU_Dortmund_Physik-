% Diskussion L1
% Relative Fehler L1:
% rel. T+ L1:               0,0442152264703583     0.0437
% rel. omega+ L1:           0,04234302024096516     0,0424
% rel. T- L1:               0,04421522647035821     0,0435
% rel. omega- L1:           0,042343020240964944     0.0425
% rel. T_S L1:              0,016041803428131787 
% rel. omega_S L1 test2:    0,016303338377607397 (ergibt eig am meisten Sinn, mit 2pi*(1/T), aber mit abs(omega_theo))
% Diskussion L2   
% Relative Fehler L2:   
% rel. T+ L2:               0,033108177656174805
% rel. omega+ L2:           0,0342418633512878
% rel. T- L2:               0,03310817765617467
% rel. omega- L2:           0,034241863351287635
% rel. T_S L2:              0,010945563805828885
% rel. omega_S L2 test2:    0,010827055578168007 (ergibt eig am meisten Sinn, mit 2pi*(1/T), aber mit abs(omega_theo))
\section{Diskussion}
\label{sec:Diskussion}
Die relative Abweichunge zwischen dem theoretischen und dem experimentellen Wert wird bestimmt durch
$$\text{rel. Abweichung} = \frac{|\text{exp. Wert} - \text{theo. Wert}|}{\text{theo. Wert}}\,.$$
\subsection{Kurzes Pendel}
Die relative Abweichung der Schwingungsdauer und der Schwingungsfrequenz der gleichphasigen Schwingung des kurzen Pendels sind 
in der Tabelle (\ref{tab:AbweichgungGleichphasig_L1}) aufgeführt. 
\begin{table}[H]
    \centering
    \caption{Relative Abweichungen der Schwingungsdauer und -frequenz der gleichphasigen Schwingung bei einer Länge von $32,5\,\unit{\centi\meter}$.}
    \label{tab:AbweichgungGleichphasig_L1}
    \begin{tblr}{colspec={c| c c}}
        \toprule
                    & $T_+$    & $\omega_+$\\
        \midrule
        exp. Wert   & $\left( 1,194 \pm 0,004 \right)\,\unit{\second}$      & $ \left( 5,261 \pm 0,017 \right)\,\unit[per-mode=fraction]{\per \second}$\\
        theo. Wert  & $1,144\,\unit{\second}$       & $5,494\,\unit[per-mode=fraction]{\per \second}$\\
        \midrule
        rel. Abweichung & $4,42\%$     & $4,23\%$ \\
        \bottomrule
    \end{tblr}
  \end{table}
Hierbei fällt auf, dass die Theoriewerte nicht mit den experimentellen Werten und deren Messunsicherheiten überereinstimmen. Dennoch ist die 
relative Abweichung von der Schwingungsdauer mit $4,42\%$ und von der Schwingungsfrequenz mit $4,23\%$ gering. Eine mögliche Erklärung für diese
Abweichung könnte eine ungenaue Messung der Pendellänge sein, welche zur Berechnung des theoretischen Werts verwendet wurde.
% \begin{align*}
%     T_{+,\text{exp.}, 1} &= \left( 1,194 \pm 0,004 \right)\,\unit{\second}\\
%     T_{+, \text{theo.}, 1} &= 1,144\,\unit{\second}
% \end{align*}
% Abweichung: $4,42\%$s
% \begin{align*}
%     \omega_{+,\text{exp.}, 1} &=  \left( 5,261 \pm 0,017 \right)\,\unit[per-mode=fraction]{\per \second}\\
%     \omega_{+, \text{theo.}, 1} &= 5,494\,\unit[per-mode=fraction]{\per \second}
% \end{align*}
% Abweichung: $4,23\%$

Gegenphasige Schwingung:\\
\begin{align*}
    T_{-,\text{exp.}, 1} &= \left(1,031 \pm 0,008 \right)\, \unit{\second}\\
    T_{-,\text{theo.},1} &= \left( 0,988 \pm 0,008 \right)\, \unit{\second}
\end{align*}
Abweichung: $4,42\%$
\begin{align*}
    \omega_{-,\text{exp.},1} &= \left( 6,09 \pm 0,05 \right)\, \unit[per-mode=fraction]{\per \second}\\
    \omega_{-,\text{theo.},1} &= \left( 6,36 \pm 0,05 \right)\,\unit[per-mode=fraction]{\per \second}
\end{align*}
Abweichung: $4,23\%$

Gekoppelte Schwingung:\\
\begin{align*}
    T_{\text{S,exp.},1} &= \left( 7,119 \pm 0,009 \right)\, \unit{\second}\\
    T_{\text{S,theo.},1} &= \left( 7,2 \pm 0,4 \right)\, \unit{\second}
\end{align*}
Abweichung: $1,60å\%$
\begin{align*}
    \omega_{\text{S,exp.},1} &= \left( 0,8826 \pm 0,0011 \right)\unit[per-mode=fraction]{\per \second}\\
    \omega_{\text{S,theo.},1} &= \left( 0,87 \pm 0,05 \right)\unit[per-mode=fraction]{\per \second}
\end{align*}
Abweichung: $1,63\%$
%
%
\subsection{Langes Pendel}
Gleichphasige Schwingung:\\
\begin{align*}
    T_{+,\text{exp.},2} &= \left( 1,567 \pm 0,010 \right)\, \unit{\second}\\
    T_{+,\text{theo.},2} &= 1,621\, \unit{\second}
\end{align*}
Abweichung: $3,31\%$
\begin{align*}
    \omega_{+,\text{exp.},2} &= \left( 4,009 \pm 0,025 \right)\, \unit[per-mode=fraction]{\per\second}\\
    \omega_{+,\text{theo.},2} &= 3,876 \, \unit[per-mode=fraction]{\per\second}
\end{align*}
Abweichung: $3,42\%$

Gegenphasige Schwingung:\\
\begin{align*}
    T_{-,\text{exp.},2} &= \left(1,420 \pm 0,006\right)\,\unit{\second}\\
    T_{-,\text{theo.},2} &= \left( 1,468 \pm 0.011 \right)\,\unit{\second}
\end{align*}
Abweichung: $3,31\%$
\begin{align*}
    \omega_{-,\text{exp.},2} &= \left(4,426 \pm 0,017 \right)\, \unit[per-mode=fraction]{\per\second}\\
    \omega_{-,\text{theo.},2} &= \left( 4,279 \pm 0,031 \right)\, \unit[per-mode=fraction]{\per\second}
\end{align*}
Abweichung: $3,42\%$

Gekoppelte Schwingung:\\
\begin{align*}
    T_{\text{S,exp.},2} &= \left( 15,741 \pm 0,013 \right)\,\unit{\second}\\
    T_{\text{S,theo.},2} &= \left( 15,6 \pm 1,2 \right)\,\unit{\second}
\end{align*}
Abweichung: $1,10\%$
\begin{align*}
    \omega_{\text{S,exp.},2} &= \left( 0,39917 \pm 0,00033  \right)\, \unit[per-mode=fraction]{\per\second}\\
    \omega_{\text{S,exp.},2} &= \left( 0,404 \pm 0,031  \right)\, \unit[per-mode=fraction]{\per\second}
\end{align*}
Abweichung: $1,08\%$
