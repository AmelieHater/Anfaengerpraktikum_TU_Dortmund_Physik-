% Diskussion L1
% Relative Fehler L1:
% rel. T+ L1:               0,0442152264703583     0.0437
% rel. omega+ L1:           0,04234302024096516     0,0424
% rel. T- L1:               0,04421522647035821     0,0435
% rel. omega- L1:           0,042343020240964944     0.0425
% rel. T_S L1:              0,016041803428131787 
% rel. omega_S L1 test2:    0,016303338377607397 (ergibt eig am meisten Sinn, mit 2pi*(1/T), aber mit abs(omega_theo))
% Diskussion L2   
% Relative Fehler L2:   
% rel. T+ L2:               0,033108177656174805
% rel. omega+ L2:           0,0342418633512878
% rel. T- L2:               0,03310817765617467
% rel. omega- L2:           0,034241863351287635
% rel. T_S L2:              0,010945563805828885
% rel. omega_S L2 test2:    0,010827055578168007 (ergibt eig am meisten Sinn, mit 2pi*(1/T), aber mit abs(omega_theo))
\section{Diskussion}
\label{sec:Diskussion}
Die relative Abweichunge zwischen dem theoretischen und dem experimentellen Wert wird bestimmt durch
$$\text{rel. Abweichung} = \frac{|\text{exp. Wert} - \text{theo. Wert}|}{\text{theo. Wert}}\,.$$
Die relative Abweichung der Schwingungsdauer und der Schwingungsfrequenz der gleichphasigen Schwingung des kurzen Pendels sind 
in der Tabelle (\ref{tab:AbweichgungGleichphasig_L1}) und die des langen Pendels in Tabelle (\ref{tab:AbweichgungGleichphasig_L2}) aufgeführt. 
\begin{table}[H]
    \centering
    \caption{Relative Abweichungen der Schwingungsdauer und -frequenz der gleichphasigen Schwingung bei einer Länge von $32,5\,\unit{\centi\meter}$.}
    \label{tab:AbweichgungGleichphasig_L1}
    \begin{tblr}{colspec={c| c c}}
        \toprule
                    & $T_+$    & $\omega_+$\\
        \midrule
        exp. Wert   & $\left( 1,194 \pm 0,004 \right)\,\unit{\second}$      & $ \left( 5,261 \pm 0,017 \right)\,\unit[per-mode=fraction]{\per \second}$\\
        theo. Wert  & $1,144\,\unit{\second}$       & $5,494\,\unit[per-mode=fraction]{\per \second}$\\
        \midrule
        rel. Abweichung & $4,42\,\%$     & $4,23\,\%$ \\
        \bottomrule
    \end{tblr}
  \end{table}

  \begin{table}[H]
    \centering
    \caption{Relative Abweichungen der Schwingungsdauer und -frequenz der gleichphasigen Schwingung bei einer Länge von $65,3\,\unit{\centi\meter}$.}
    \label{tab:AbweichgungGleichphasig_L2}
    \begin{tblr}{colspec={c| c c}}
        \toprule
                    & $T_+$    & $\omega_+$\\
        \midrule
        exp. Wert   & $\left( 1,567 \pm 0,010 \right)\,\unit{\second}$      & $ \left( 4,009 \pm 0,025 \right)\,\unit[per-mode=fraction]{\per \second}$\\
        theo. Wert  & $1,621\,\unit{\second}$       & $3,876\,\unit[per-mode=fraction]{\per \second}$\\
        \midrule
        rel. Abweichung & $3,31\,\%$     & $3,42\,\%$ \\
        \bottomrule
    \end{tblr}
  \end{table}
Hierbei fällt auf, dass die Theoriewerte nicht mit den experimentellen Werten und deren Messunsicherheiten überereinstimmen. Dennoch ist die 
relative Abweichung von der Schwingungsdauer mit $4,42\,\%$ (bzw. $3,31\,\%$ beim langen Pendel) und von der Schwingungsfrequenz mit 
$4,23\,\%$ (bzw. $3,42\,\%$ beim langen 
Pendel) gering. Eine mögliche Erklärung für diese
Abweichung könnte eine ungenaue Messung der Pendellänge sein, welche zur Berechnung des theoretischen Werts verwendet wurde.
Außerdem könnte ein Messfehler durch die Reaktionszeit der messenden Person zustande kommen. 

Die Messergebnisse, Theorieergebnisse und Abweichungen der gegenphasige Schwingung sind in Tabelle (\ref{tab:AbweichgungGegenphasig_L1}) für 
das kurze Pendel und in Tabelle (\ref{tab:AbweichgungGegenphasig_L2}) für 
das lange Pendel aufgeführt. 
\begin{table}[H]
    \centering
    \caption{Relative Abweichungen der Schwingungsdauer und -frequenz der gegenphasigen Schwingung bei einer Länge von $32,5\,\unit{\centi\meter}$.}
    \label{tab:AbweichgungGegenphasig_L1}
    \begin{tblr}{colspec={c| c c}}
        \toprule
                    & $T_-$    & $\omega_-$\\
        \midrule
        exp. Wert   & $\left(1,031 \pm 0,008 \right)\, \unit{\second}$      & $ \left( 6,09 \pm 0,05 \right)\, \unit[per-mode=fraction]{\per \second}$\\
        theo. Wert  & $\left( 0,988 \pm 0,008 \right)\, \unit{\second}$       & $\left( 6,36 \pm 0,05 \right)\,\unit[per-mode=fraction]{\per \second}$\\
        \midrule
        rel. Abweichung & $4,42\,\%$     & $4,23\,\%$ \\
        \bottomrule
    \end{tblr}
  \end{table}

  \begin{table}[H]
    \centering
    \caption{Relative Abweichungen der Schwingungsdauer und -frequenz der gegenphasigen Schwingung bei einer Länge von $65,3\,\unit{\centi\meter}$.}
    \label{tab:AbweichgungGegenphasig_L2}
    \begin{tblr}{colspec={c| c c}}
        \toprule
                    & $T_-$    & $\omega_-$\\
        \midrule
        exp. Wert   & $\left(1,420 \pm 0,006 \right)\, \unit{\second}$      & $ \left( 4,426 \pm 0,017 \right)\, \unit[per-mode=fraction]{\per \second}$\\
        theo. Wert  & $\left( 1,468 \pm 0.011 \right)\, \unit{\second}$       & $\left( 4,279 \pm 0,031 \right)\,\unit[per-mode=fraction]{\per \second}$\\
        \midrule
        rel. Abweichung & $3,31\,\%$     & $3,42\,\%$ \\
        \bottomrule
    \end{tblr}
  \end{table}

Zusätzlich zu den genannten Gründen für Messunsicherheiten muss bei der gegensinnigen
Schwingung auch die Messunsicherheit der Kopplungskonstante $K$ berücksichtigt werden,
die durch die Experimentalwerte bestimmt wird. 
Dadurch, dass auch bei Berechnung der Theoriewerte die experimentell berechnete Kopplungskonstante $K$ verwendet wird, 
könnte es zu einer geringeren relativen Abweichung zwischen Experimentalwert und Theoriewert gekommen sein, als eigentlich 
realistisch wäre. Dieser Umstand könnte die geringere relative Abweichung der Werte der gegensinnigen Schwingung erklären, im Vergleich zu den Werten
der gleichsinnigen Schwingung. Bei diesen Werten wird die Kopplungskonstante $K$ nicht zur Berechnung verwendet. Diese Überlegung trifft genauso auf die 
Messwerte der gekoppelten Schwingung zu. Die experimentelle Messwerte, die berechneten Theoriewerte und die dazugehörigen Abweichungen der gekoppelten 
Schwingungen sind für das kurze Pendel in Tabelle (\ref{tab:AbweichgungGekoppelt_L1}) und für das lange Pendel in Tabelle (\ref{tab:AbweichgungGekoppelt_L2}) aufgeführt. 

\begin{table}[H]
    \centering
    \caption{Relative Abweichungen der Schwebungsdauer und -frequenz der gekoppelten Schwingung bei einer Länge von $32,5\,\unit{\centi\meter}$.}
    \label{tab:AbweichgungGekoppelt_L1}
    \begin{tblr}{colspec={c| c c}}
        \toprule
                    & $T_{\text{S}}$    & $\omega_{\text{S}}$ & \\
        \midrule
        exp. Wert   & $\left(7,119 \pm 0,009 \right)\, \unit{\second}$      & $ \left( 0,8826 \pm 0,0011 \right)\, \unit[per-mode=fraction]{\per \second}$\\
        theo. Wert  & $\left( 7,2 \pm 0,4 \right)\, \unit{\second}$       & $\left( 0,87 \pm 0,05 \right)\,\unit[per-mode=fraction]{\per \second}$\\
        \midrule
        rel. Abweichung & $1,60\,\%$     & $1,63\,\%$ \\
        \bottomrule
    \end{tblr}
  \end{table}
  \begin{table}[H]
    \centering
    \caption{Relative Abweichungen der Schwebungsdauer und -frequenz der gekoppelten Schwingung bei einer Länge von $65,3\,\unit{\centi\meter}$.}
    \label{tab:AbweichgungGekoppelt_L2}
    \begin{tblr}{colspec={c| c c}}
        \toprule
                    & $T_{\text{S}}$    & $\omega_{\text{S}}$ & \\
        \midrule
        exp. Wert   & $\left(15,741 \pm 0,013 \right)\, \unit{\second}$      & $ \left( 0,39917 \pm 0,00033 \right)\, \unit[per-mode=fraction]{\per \second}$\\
        theo. Wert  & $\left( 15,6 \pm 1,2 \right)\, \unit{\second}$       & $\left( 0,404 \pm 0,031 \right)\,\unit[per-mode=fraction]{\per \second}$\\
        \midrule
        rel. Abweichung & $1,10\,\%$     & $1,08\,\%$ \\
        \bottomrule
    \end{tblr}
  \end{table}

Die Abweichung ist bei diesen Messwerten auffällig gering, sodass die experimentell bestimmten Werte mit denen der Theorie innerhalb der Messunsicherheit übereinstimmen. 
Ein Grund dafür könnte die lange Schwebungsdauer sein, da der Messfehler der Reaktionszeit klein im Vergleich zum gemessenen Zeitraum ist.  
%Gekoppelte Schwingung:\\
% \begin{align*}
%     T_{\text{S,exp.},1} &= \left( 7,119 \pm 0,009 \right)\, \unit{\second}\\
%     T_{\text{S,theo.},1} &= \left( 7,2 \pm 0,4 \right)\, \unit{\second}
% \end{align*}
%Abweichung: $1,60\%$
% \begin{align*}
%     \omega_{\text{S,exp.},1} &= \left( 0,8826 \pm 0,0011 \right)\unit[per-mode=fraction]{\per \second}\\
%     \omega_{\text{S,theo.},1} &= \left( 0,87 \pm 0,05 \right)\unit[per-mode=fraction]{\per \second}
% \end{align*}
%Abweichung: $1,63\%$
%
%
%\subsection{Langes Pendel}
%Gleichphasige Schwingung:\\
%\begin{align*}
    %T_{+,\text{exp.},2} &= \left( 1,567 \pm 0,010 \right)\, \unit{\second}\\
    %T_{+,\text{theo.},2} &= 1,621\, \unit{\second}
%\end{align*}
%Abweichung: $3,31\%$
%\begin{align*}
    %\omega_{+,\text{exp.},2} &= \left( 4,009 \pm 0,025 \right)\, \unit[per-mode=fraction]{\per\second}\\
    %\omega_{+,\text{theo.},2} &= 3,876 \, \unit[per-mode=fraction]{\per\second}
%\end{align*}
%Abweichung: $3,42\%$

%Gegenphasige Schwingung:\\
%\begin{align*}
%    T_{-,\text{exp.},2} &= \left(1,420 \pm 0,006\right)\,\unit{\second}\\
%    T_{-,\text{theo.},2} &= \left( 1,468 \pm 0.011 \right)\,\unit{\second}
%\end{align*}
% Abweichung: $3,31\%$
% \begin{align*}
%     \omega_{-,\text{exp.},2} &= \left(4,426 \pm 0,017 \right)\, \unit[per-mode=fraction]{\per\second}\\
%     \omega_{-,\text{theo.},2} &= \left( 4,279 \pm 0,031 \right)\, \unit[per-mode=fraction]{\per\second}
% \end{align*}
% Abweichung: $3,42\%$

%Gekoppelte Schwingung:\\
% \begin{align*}
%     T_{\text{S,exp.},2} &= \left( 15,741 \pm 0,013 \right)\,\unit{\second}\\
%     T_{\text{S,theo.},2} &= \left( 15,6 \pm 1,2 \right)\,\unit{\second}
% \end{align*}
% Abweichung: $1,10\%$
% \begin{align*}
%     \omega_{\text{S,exp.},2} &= \left( 0,39917 \pm 0,00033  \right)\, \unit[per-mode=fraction]{\per\second}\\
%     \omega_{\text{S,exp.},2} &= \left( 0,404 \pm 0,031  \right)\, \unit[per-mode=fraction]{\per\second}
% \end{align*}
% Abweichung: $1,08\%$
