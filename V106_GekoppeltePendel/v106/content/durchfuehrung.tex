\section{Aufbau}
\label{sec:Aufbau}
Der Versuch wird mithilfe zweier Fadenpendel durchgeführt, die an einer Wand befestigt sind. Da beide Pendel identisch sind wird im Folgenden 
nur der Aufbau eines Pendel beschrieben. 
Das Pendel besteht aus einem Metallstab, der an einer reibungsarmen Aufhängung befestigt ist, und einer Masse $(m = 1 \,\unit{\kilo\gram})$, 
die entlang des Stabes verschoben werden kann. Die Pendellänge ist die Länge von der Aufhängung bis zum Mittelpunkt der Masse und wird mithilfe eines 
Maßbandes gemessen.
Die Pendel sind über eine Feder verbunden. Die Schwingdauer werden mithilfe einer Stopuhr gemessen. 

\section{Durchführung}
\label{sec:Durchführung}
Dann bei beiden Pendeln dieselbe Höhe der Massen eingestellt und die daraus entstehende Länge der Pendel notiert. 
Danach wird die Kopplungsfeder entfernt und 10 Mal die Schwingdauer $T$ der beiden Pendel bei gleicher Auslenkung gemessen. 