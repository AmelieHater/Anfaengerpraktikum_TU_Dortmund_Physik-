\section{Zielsetzung}
\label{sec:Zielsetzung}
Das Ziel des Versuchs ist die Schwingdauer, Schwebungsdauer und die Kopplungskonstante $K$ gekoppelter Pendel bei verschiedener Schwingungsformen zu bestimmen. 
Es werden gleichsinnige Schwingungen, gegensinnige Schwingung und gekoppelte Schwingung betrachtet.  

\section{Theorie}
\label{sec:Theorie}
Zunächst wird ein einzelnens ungekoppeltes Fadenpendel mit Fadenlänge $l$ betrachtet. 
Wird die Masse am Stab aus der Ruhelage ($\varphi = 0$) ausgelenkt, wirkt 
die Gravitationskraft $\vec{F} = m \cdot \vec{g}$ als Rückstellkraft der Bewegung entgegen. Dies bewirkt ein Drehmoment 
$M = D_{\text{P}} \cdot \varphi$ auf das Pendel. $\varphi$ ist dabei der Auslenkungswinkel aus der Ruhelage und $D_{\text{P}}$ die 
Winkelrichtgröße des Pendels. Die das System beschreibende Differentialgleichung lautet für die angenommene Kleinwinkelnäherung 
$\sin(\varphi) = \varphi$ für $\varphi \leq 10°$
\begin{equation*}
    J \cdot \ddot{\varphi} + D_{\text{P}} \cdot \varphi = 0\,.
\end{equation*}
$J$ bezeichnet das Trägheitsmoment des Pendels. Diese Differentialgleichung ist die des harmonischen Oszillators und beschreibt eine harmonische Schwingung 
mit Schwingfrequenz 

\begin{equation*}
 \omega = \sqrt{\frac{D_{\text{P}}}{J}} = \sqrt{\frac{g}{l}}\, .
\end{equation*}
Die Periodendauer $T$ und die Schwingfrequenz $\omega$ sind verbunden durch 
\begin{align}
    \omega &= \frac{2 \pi}{T} \,\,\text{bzw.} \label{eqn:VerbindungOmegaT}\\
    \Leftrightarrow T &= \frac{2 \pi}{\omega} \label{eqn:VerbindungTOmega}\, .
\end{align}
\\
Zwei identische Pendel, die durch eine Feder mit Federkonstante $K$ gekoppelt sind, haben eine andere Bewegungsgleichung. 
Auf jeden Pendel wirkt dann ein zusätzliches Drehmoment. Auf das eine Pendel wirkt $M_1 = D_{\text{F}} \cdot (\varphi_2 - \varphi_1)$ 
und auf das andere wirkt $M_2 = D_{\text{F}} \cdot (\varphi_1 - \varphi_2)$. 
Durch diese Kopplung entstehen die beiden gekoppelten Differentialgleichungungen
\begin{align*}
J \ddot{\varphi_1} + D \varphi_1 &= D_{\text{F}} \cdot \left( \varphi_{2} - \varphi_{1} \right) \\
\Leftrightarrow J \ddot{\varphi_2} + D \varphi_2 &= D_{\text{F}} \cdot \left( \varphi_{1} - \varphi_{2} \right) \, ,
\end{align*}
die das System vollkommen beschreiben.
Mithilfe eines geeigneten Winkels können die Gleichungen entkoppelt werden, sodass sich das System als Überlagerung von zwei Eigenschwingungen darstellen lässt. 
Diese Eigenschwingungen sind harmonische Schwingungen mit Frequenz $\omega_{1}$ und $\omega_{2}$. 
Je nach Auslenkungswinkel $\alpha_1$ und $\alpha_2$ der Pendel beim Zeitpunkt $t = 0$ entstehen verschiedene Schwingungsarten. 
\\
Ist $\alpha_1 = \alpha_2$, schwingt das Pendel gleichsinnig. Bei dieser Bewegung übt die Feder keine Kraft aus, da sie durch die gleich große Auslenkung beider
Pendel, weder gestreckt noch gestaucht wird. Beide Pendel schwingen daher gleich und identisch zu einem ungekoppelten Fadenpendel, welches um 
denselben Winkel ausgelenkt wurde. Daher ist die Schwingfrequenz $\omega_+$ bei diesen konkreten Anfangsbedingungen 
\begin{equation}
    \omega_+ = \sqrt{\frac{g}{l}}\, ,
    \label{eqn:OmegaGleichsinnig}
\end{equation}
dieselbe wie bei einem ungekoppelten Fadenpendel. \\
Bei einer gegensinnigen Schwingung ist $\alpha_1 = - \alpha_2$. 
Die Kopplungsfeder übt dann eine gleich große Kraft auf die Pendel aus. Die Kraft zeigt immer zur Mitte zwischen den beiden Pendeln. 
Die Schwingung bei diesen Anfangsbedingungen ist symmetrisch.
Die Frequenz $\omega_-$ wird durch
\begin{equation}
    \omega_- = \sqrt{\frac{g}{l} \cdot \frac{1 + \text{K}}{1 - \text{K}}}
    \label{eqn:OmegaGegensinnig}
\end{equation}
berechnet. 
Falls nur ein Pendel aus der Ruhelage ausgelenkt wird, heißt die entstandene Bewegung gekoppelte Schwingung. Dann gilt für 
die Auslenkungen $\alpha_1 = 0$ und $\alpha_2 \neq 0$. 
Für $t > 0$ führt das zweite Pendel eine Schwingung aus, ähnlich zu der eines ungekoppelten Fadenpendels. Im fortschreitender Zeit wird die Pendelbewegung des 
zweiten Pendels langsamer bis dieses vollkommen still steht. Während die Schwiungung des zweiten Pendels langsamer wird, nimmt die des ersten Pendels zu bis es die 
maximale Auslenkung erreicht, wenn das zweite Pendel stillsteht. Dieses Muster wiederholt sich dauerhaft. Diese Bewegung kommt dadurch zustande, dass 
die Feder für eine Energieübertragung zwischen den Pendeln sorgt. Die zwischen zwei Stillständen eines Pendels vergangene Zeit nennt man Schwebung.
Die Schwebungsfrequenz $\omega_{\text{S}}$ wird durch 
\begin{equation}
    \omega_{\text{S}} =  \omega_+ - \omega_-
    \label{eqn:OmegaSchwebung}
\end{equation}
berechnet. Die sich daraus ergebene Schwebedauer $T_{\text{S}}$ ist
\begin{equation}
    T_{\text{S}} =  \frac{T_+ \cdot T_-}{T_+ - T_-}\,.
    \label{eqn:TSchwebung}
\end{equation}
 
%Zwei identische durch eine Feder mit Federkonstante $K$ gekoppelte Fadenpendel
\subsection{Vorbereitungsaufgaben}
\label{sec:Vorbereitungsaufgaben}
\textbf{Wann spricht man von einer harmomischen Schwingung?}\\
Eine Schwingung ist dann harmonisch, wenn die rücktreibende Kraft linear ist (d.h. proportional zur Auslenkung) und sich die Bahnkurve als 
Cosinus bzw. Sinus beschrieben werden kann, d.h. die Projektion einer gleichförmigen Kreisbewegung ist. 
\\
\\
\textbf{Wie weit kann man ein Fadenpendel von $l = 70 \,\,\unit{\centi\meter}$ auslenken, damit die Kleinwinkelnäherung noch gilt?}\\
Bei der Kleinwinkelnäherung ist es ausschlaggebend, dass der Winkel $\alpha$ möglichst keinen Wert über $10°$ annimmt, 
damit die Taylorentwicklung des $\sin(\alpha)$ mit Entwicklungspunkt $\alpha = 0$ noch gültig ist. 
Der Winkel $\alpha$ im Fadenpendelsystem kann durch
\begin{align*}
    \sin(\alpha) &= \frac{x}{l} \\
    \Leftrightarrow x &= \sin(\alpha) \cdot l
\end{align*}
ausgdrückt werden. $x$ ist dabei die Auslenkung aus der Ruhelage, senkrecht zur Ruhelage. \\
Einsetzen von $\alpha = 10°$ und $l = 70 \,\, \unit{\centi\meter}$ ergibt eine maximale Auslenkung von $x = 12,16 \,\, \unit{\centi\meter}$. 