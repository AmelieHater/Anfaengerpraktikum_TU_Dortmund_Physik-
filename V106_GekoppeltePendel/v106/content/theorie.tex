\section{Zielsetzung}
\label{sec:Zielsetzung}
Das Ziel des Versuchs ist die Schwingdauer, Schwebungsdauer und die Kopplungskonstante $K$ gekoppelter Pendel bei verschiedener Schwingungsformen zu bestimmen. 
Es werden gleichsinnige Schwinung, gegensinnige Schwingung und gekoppelte Schwingung betrachtet.  

\section{Theorie}
\label{sec:Theorie}
Zunächst wird ein einzelnens ungekoppeltes Fadenpendel mit Fadenlänge $l$ betrachtet. 
Wird die Masse am Stab aus der Ruhelage ($\varphi = 0$) ausgelenkt, wirkt 
die Gravitationskraft $\vec{F} = m \cdot \vec{g}$ als Rückstellkraft der Bewegung entgegen. Dies bewirkt ein Drehmoment 
$M = D_{\text{P}} \cdot \varphi$ auf das Pendel. $\varphi$ ist dabei der Auslenkungswinkel aus der Ruhelage und $D_{\text{P}}$ die 
Winkelrichtgröße des Pendels. Die das System beschreibende Differentialgleichung lautet für die angenommene Kleinwinkelnäherung 
$\sin(\varphi) = \varphi$ für $\varphi \leq 10°$
\begin{equation}
    J \cdot \ddot{\varphi} + D_{\text{P}} \cdot \varphi = 0\,.
\end{equation}
$J$ bezeichnet das Trägheitsmoment des Pendels. Diese Differentialgleichung ist die des harmonischen Oszillators und beschreibt eine harmonische Schwingung 
mit Schwingfrequenz 

\begin{equation}
 \omega = \sqrt{\frac{D_{\text{P}}}{J}} = \sqrt{\frac{g}{l}}\, .
\end{equation}
\\
\\
%Zwei identische durch eine Feder mit Federkonstante $K$ gekoppelte Fadenpendel

\subsection{Vorbereitungsaufgaben}
\label{sec:Vorbereitungsaufgaben}
\textbf{Wann spricht man von einer harmomischen Schwingung?}\\
Eine Schwingung ist dann harmonisch, wenn die rücktreibende Kraft linear ist (d.h. proportional zur Auslenkung) und sich die Bahnkurve als 
Cosinus bzw. Sinus beschrieben werden kann, d.h. die Projektion einer gleichförmigen Kreisbewegung ist. 
\\
\\
\textbf{Wie weit kann man ein Fadenpendel von $l = 70 \,\,\unit{\centi\meter}$ auslenken, damit die Kleinwinkelnäherung noch gilt?}\\
Bei der Kleinwinkelnäherung ist es ausschlaggebend, dass der Winkel $\alpha$ möglichst keinen Wert über $10°$ annimmt, 
damit die Taylorentwicklung des $\sin(\alpha)$ mit Entwicklungspunkt $\alpha = 0$ noch gültig ist. 
Der Winkel $\alpha$ im Fadenpendelsystem kann durch
\begin{align}
    \sin(\alpha) &= \frac{x}{l} \\
    \Leftrightarrow x &= \sin(\alpha) \cdot l
\end{align}
ausgdrückt werden. $x$ ist dabei die Auslenkung aus der Ruhelage, senkrecht zur Ruhelage. \\
Einsetzen von $\alpha = 10°$ und $l = 70 \,\, \unit{\centi\meter}$ ergibt eine maximale Auslenkung von $x = 12,16 \,\, \unit{\centi\meter}$. 