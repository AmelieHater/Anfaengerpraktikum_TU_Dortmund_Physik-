\section{Zielsetzung}
\label{sec:Zielsetzung}

\section{Theorie}
\label{sec:Theorie}

\subsection{Vorbereitungsaufgaben}
\label{sec:Vorbereitungsaufgaben}
\textbf{Wann spricht man von einer harmomischen Schwingung?}\\
Eine Schwingung ist dann harmonisch, wenn die rücktreibende Kraft linear ist (d.h. proportional zur Auslenkung) und sich die Bahnkurve als 
Cosinus bzw. Sinus beschrieben werden kann, d.h. die Projektion einer gleichförmigen Kreisbewegung ist. 
\\
\\
\textbf{Wie weit kann man ein Fadenpendel von $l = 70 \,\,\unit{\centi\meter}$ auslenken, damit die Kleinwinkelnäherung noch gilt?}\\
Bei der Kleinwinkelnäherung ist es ausschlaggebend, dass der Winkel $\alpha$ möglichst keinen Wert über $5°$ annimmt, 
damit die Taylorentwicklung des $\sin(\alpha)$ mit Entwicklungspunkt $\alpha = 0$ noch gültig ist. 
Der Winkel $\alpha$ im Fadenpendelsystem kann durch
\begin{align}
    \sin(\alpha) &= \frac{x}{l} \\
    \Leftrightarrow x &= \sin(\alpha) \cdot l
\end{align}
ausgdrückt werden. $x$ ist dabei die Auslenkung aus der Ruhelage, senkrecht zur Ruhelage. \\
Einsetzen von $\alpha = 5°$ und $l = 70 \,\, \unit{\centi\meter}$ ergibt eine maximale Auslenkung von $x = 6,10 \,\, \unit{\centi\meter}$. 