\section{Zielsetzung}
\label{sec:Zielsetzung}
In diesem Versuch wird die Suszeptibilität stark paramagnetischer Materialien 
mithilfe einer Brückenschaltung untersucht. 
\nocite{anleitungV606}
\section{Theorie}
\label{sec:Theorie}
In Materie wird die magnetische Flussdichte $\vec{B}$ durch die magnetische Feldstärke
$\vec{H}$, die Induktionskonstante $\mu _0$ und die Magnetisierung $\vec{M}$ wie folgt beschrieben
\begin{equation}
    \vec{B} = \mu _0 \vec{H} + \vec{M}\,.
    \label{eqn:magnetischeFlussdichte}
\end{equation}
Dabei hängt $\vec{M}$ mit $\vec{H}$ durch 
\begin{equation}
    \vec{M} = \mu _0\, \chi\, \vec{H}
    \label{eqn:Magnetisierung}
\end{equation}
zusammen. Hierbei beschreibt $\chi$ die Suszeptibilität, welche keine Konstante ist, sondern von $\vec{H}$ und der Temperatur $T$ abhängt.
Der Diamagnetismus tritt für alle Atome auf, weil durch ein von außen angelegtes Magnetfeld ein magnetischer Moment induziert wird. Dadurch
entsteht ein induziertes Magnetfeld, was dem äußeren Magnetfeld entgegengerichtet ist. Daher gilt beim Diamagnetimus für die Suszeptibilität
$\chi < 0 $. Anders als beim Diamagnetismus, tritt der Paramagnetismus nur bei Atomen, Ionen oder Molekülen mit einem nicht veschwindenen Drehimpuls auf.
Dieser entsteht durch die relativ zum äußeren Magnetfeld ausgerichteten magnetischen Momente, die mit dem Drehimpuls gekoppelt sind. Zusätzlich ist der
Paramagnetismus im Vergleich zum Diamagnetimus temperaturabhängig, da die Ausrichtung der magnischten Moment durch
die thermische Bewegung gestört wird. Bei einem nicht zu starken äußeren Magnetfeld auf die Atome gilt für den Gesamtdrehimpuls $\vec{J}$
\begin{equation}
    \vec{J} = \vec{L} + \vec{S}\,.
    \label{eqn:Gesamtdrehimpuls}
\end{equation}
Diese Gleichung wird ebenfalls als LS-Kopplung bezeichnet, wobei $\vec{L} = \sum{\vec{l}_{\text{i}}}$ den Gesamtbahndrehimpuls und 
$\vec{S} = \sum{\vec{s}_{\text{i}}}$ den Gesamtspin beschreibt.
Die zugehörigen magnetischen Momente zu dem Drehimpuls $\vec{L}$ und dem Spin $\vec{S}$ lauten
\begin{align}
    \vec{\mu _{\text{L}}} &= - \frac{\mu_{\text{B}}}{\hbar}\,\vec{L}\quad\text{und} \label{eqn:magnetischerMommentDrehimpuls} \\
    \vec{\mu _ {\text{S}}} &= -g_{\text{S}}\frac{\mu_{\text{B}}}{\hbar}\,\vec{S}\,. \label{eqn:magnetischerMomentSpin}
\end{align}
$\hbar$ ist das reduzierte Planksche Wirkungsquantum, $g_{\text{S}}$ ist das gyromagnetische Verhältnis des freien Elektrons und 
\begin{equation}
    \mu _{\text{B}} = \frac{1}{2} \frac{e_0}{m_0}\,\hbar
    \label{eqn:BohrscheMagneton}
\end{equation}
ist das Bohrsche Magneton, wobei $e_0$ die Ladung und $m_0$ die Ruhemasse des Elektrons sind. 
In der Quantenmechanik wird $g_{\text{S}} \approx 2$ genähert, wodurch mit dem Landé-Faktor
\begin{equation}
    g_{\text{L}} = \frac{3\,J \left(J+1\right) + \big(S\left(S +1\right) -L\left(L+1\right)\big)}{2\,J\left(J+1\right)} 
    \label{eqn:LandeFaktor}
\end{equation}
für den Betrag des magnetischen Moments 
\begin{equation}
    \left|\vec{\mu_{\text{J}}}\right| \approx \mu_{\text{B}}\,g_{\text{J}}\sqrt{J\left(J+1\right)}
    \label{eqn:magnetischerMoment}
\end{equation}
gilt. Daraus folgt für die potentielle Energie mit der Orientierungsquantenzahl $m$
\begin{equation}
    E_{\text{m}} = \mu_{\text{B}}\,g_{\text{J}}\, m\, B\,. 
    \label{eqn:E_pot}
\end{equation}
Ein Zeeman-Effekt tritt dann auf wenn die durch die Gleichung (\ref{eqn:E_pot}) beschriebene Aufspaltung eines Energieniveaus
in $2J+1$ Unterniveaus beim Anlegen eines Feldes an eine Probe mit permanenten magnetischen Momenten statfindet.
