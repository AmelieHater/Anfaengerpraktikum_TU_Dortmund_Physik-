\section{Diskussion}
\label{sec:Diskussion}
Die Abweichungen werden mithilfe der Formel
\begin{equation*}
    \text{rel. Abweichung} = \frac{|\text{exp. Wert} - \text{theo. Wert}|}{\text{theo. Wert}}
  \end{equation*}
berechnet.
Aus Ermangelung eines Theoriewerts werden jeweils die berechneten Werte der unterschiedlichen absoluten Abstände miteinander verglichen. Zwischen dem Energieverlust 
bei einem Abstand von $4,5 \, \unit{\centi\meter}$ und dem bei einem Abstand von $6 \, \unit{\centi\meter}$ besteht eine Abweichung von $4,44 \, \%$. Zwischen 
der mittleren Reichweite bei einem absoluten Abstand von $4,5 \, \unit{\centi\meter}$ und einem Abstand $6 \, \unit{\centi\meter}$ besteht eine Abweichung von 
$3,22 \, \%$. Auffällig ist, dass die statistische Verteilung eher einer Gaußverteilung gleicht, als einer Poissonverteilung, wie die Theorie beschriebt. 
Die Abweichungen können dadurch zustande kommen, dass kein perfektes Vakuum hergestellt werden kann und daher der Wert bei einem Druck von $0$ fehlerhaft sind. 
Außerdem konnte mithilfe der Skala auf dem Barometer der Druck nicht präzise eingestellt werden. Die Abweichungen sind vor allen Dingen gering, da durch den 
Versuchsaufbau größere, experimentelle Fehler verhindert wurden. 
