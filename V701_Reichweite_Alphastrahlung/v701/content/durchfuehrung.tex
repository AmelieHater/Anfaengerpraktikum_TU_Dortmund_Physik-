\section{Versuchsaufbau}
\label{sec:Versuchsaufbau}
Der Versuch wird mithilfe der Apparatur in Abbildung \ref{fig:Messapparatur} durchgeführt. In dem Glaszylinder 
befindet sich ein $\alpha$-Präparat sowie ein Detektor, dessen Distanz einstellbar ist.  In diesem Fall wird als 
Strahlungsquelle ein \ce{Am}-Präparat verwendet. Der Detektor ist ein Halbleiter-Sperrschichtzähler, der ähnlich wie 
eine Diode, an eine Gleichspannung in Sperrrichtung angelegt ist, wodurch kein Strom durchfließt.

\section{Durchführung}
\label{sec:Durchführung}
Zunächst wird der Glaszylinder Evakuiert, indem die Belüftungsventile geschlossen werden und die Drehschieberpumpe aktiviert 
wird. Sobald der Druck bei $p \approx 0\,\unit{\milli\bar}$ liegt, wird das rote Ventil zwischen der Pumpe und dem Glaszylinder 
geschlossen und die Pumpe ausgestellt. Wenn der Druck in der Apparatur konstant bleibt, kann die Messung beginnen. Um die Reichweite 
von $\alpha$-Strahlung zu bestimmen, wird die Energieverteilung und die Zählrate der $\alpha$-Strahlung in Abhängigkeit vom Druck $p$ 
in Abständen von ca. $50\,\unit{\milli\bar}$ gemessen. Der Druck wird mithilfe des Belüftungsventils eingestellt und die Messzeit beträgt $2\,\unit{\minute}$. 
Für jede Messung wird die Position des Energiemaximums und die Gesamtzählrate notiert. 
Diese Messungen werden für zwei verschiedene Abstände zwischen dem Detektor und der $\alpha$-Strahlung durchgeführt. Anschließend wird 
eine Messreihe zur Überprüfung der Statistik des radioaktiven Zerfalls aufgenommen. Dabei wird der Glaszylinder erneut evakuiert und es 
werden 100 mal die Zerfälle pro Zeiteinheit bei einem Druck von $p = 0\,\unit{\milli\bar}$ gemessen. Die Messzeit beträgt hier $10\,\unit{\second}$.
