%Abweichung Nr. 1:
%Abweichung Chi Nr.1:  (3.17+/-0.11)e+03
%Abweichung Chi Nr.1:  (7.6+/-0.4)e+03
%Abweichung Chi Nr.1:  2812+/-12
%
%Abweichung Nr.2:  311+/-21
%Abweichung Nr.2:  (1.5+/-0.5)e+02
%Abweichung Nr.2:  347+/-9
%$\ce{Nd2O3}$
%$\ce{Gd2O3}$
%$\ce{Dy2O3}$

\section{Diskussion}
\label{sec:Diskussion}
Die Abweichungen werden mithilfe der Formel
\begin{equation*}
    \text{rel. Abweichung} = \frac{|\text{exp. Wert} - \text{theo. Wert}|}{\text{theo. Wert}}
  \end{equation*}
berechnet. Der eingestellte Wert der Güte $Q$ des Selektivverstärkers ist $100$, der durch die Messwerte ermittelte Wert ist $73,52$. Dies entspricht einer Abweichung
von $26,48 \,\%$.
Die Abweichungen der Suszeptibilität $\chi$ der verschiedenen Materialien ist auffällig hoch. Die Abweichung der 1. Methode zur Bestimmung beim Material $\ce{Gd2O3}$ ist 
$3,17 \cdot 10⁵ \, \%$, bei $\ce{Nd2O3}$ $7,6 \cdot 10⁵ \, \%$ und bei $\ce{Dy2O3}$ $2,81 \cdot \ 10⁵ \, \%$. Bei der 2. Methode sind die Fehler kleiner. Die Abweichung bei
$\ce{Gd2O3}$ ist $3,11 \cdot 10⁴ \, \%$, bei $\ce{Nd2O3}$ $1,5 \cdot 10⁴ \, \%$ und bei $\ce{Dy2O3}$ $3,47 \cdot 10⁴ \, \%$.
Eine Erklärung für die hohen Abweichungen könnte das Voltmeter sein. Die Größenordnungseinstellung des Messgerätes ließ eine genauere Auflösung nicht zu, sodass sich
der Ausschlag selbst bei größeren Veränderungen des Widerstandes nicht geändert hat. Zudem konnte die maximale Ausgangsspannung aus dem Selektivverstärker 
ebenfalls nicht genau bestimmt werden, welche als Grundlage für den 2. Teil des Versuchs verwendet wurde. Dadurch werden die experimentell bestimmten Suszeptibilitäten 
ebenfalls verfälscht. 
Außerdem konnten die minimalen Widerstände durch diesen Umstand nicht genau bestimmt werden. Zudem war keine konstante Umgebungstemperatur während der Durchführung 
gegeben, da zu Beginn die Heizung lief, dann ein Fenster geöffnet wurde und die Heizung später ausgeschaltet wurde. Dies könnte zu einer größeren Abweichung geführt 
haben, da die Temperatur entscheident für den Paramagnetismus ist. Vor allem, da die Raumtemperatur geschätzt wurde.
Außerdem wurde die Länge der Probe, die sich innerhalb der Spule befunden hat auch geschätzt, da nicht die ganze Länge der Probe hineinschiebbar war. Zudem ist die 
Zuverlässigkeit des Messgerätes in Frage zu stellen, da bei Umschalten der Größenordnung und anschließendem Zurückschalten (beispielsweise von 
$1 \, \unit{\milli\volt}$ auf $3 \, \unit{\milli\volt}$ und wieder zurück auf $1 \, \unit{\milli\volt}$) sich der angezeigte Wert 
unterschied bei derselben Größenordnung. 
