% rel. Abw. Wellenlänge = 2.5+/-1.9
% rel. Abw. Brechungsindex 1 = 0.0188+/-0.0018
% rel. Abw. Brechungsindex 2 = 0.0005+/-0.0029
\section{Diskussion}
\label{sec:Diskussion}
Die relative Abweichung der Wellenlänge beträgt $2,5 \, \%$. Die relative Abweichung des 1. Brechungsindex $n_{\text{Evak.}}$ ist $0,0188 \, \%$, die für den 2. 
Brechungsindex $n_{\text{Belüft.}}$ ist $0,0005 \, \%$. Dies sind überraschend geringe Abweichungen, vor allem da während der Durchführung mehrfach
 Probleme aufgetreten sind. So musste die Sensitivität des Schmitt-Triggers maximal gestellt werden. Trotzdessen konnten manche Interferenzmusterwechsel 
 nicht erkannt werden und die Pulsanzahl blieb konstant. Diese Messreihen wurden abgebrochen. 
 Aufgrund der anfänglichen Probleme wurde allerdings genau auf die Entwicklung der Messergebnisse geachtet, weshalb alle nicht funktionierenden Messreihen 
 aussortiert wurden. Als Reaktion auf diese nicht funktionierenden Messreihen wurde das Interferenzmuster neu justiert. Dadurch lässt sich eine hohe Präzision 
 erklären. Beim 2. Teil des Versuchs bestand kein ähnliches Problem. Die geringen Fehler dieses Teils lassen sich durch den Versuchsaufbau erklären, 
 der nicht stark modifiziert werden musste.