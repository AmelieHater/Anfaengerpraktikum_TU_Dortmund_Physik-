\section{Auswertung}
\label{sec:Auswertung}
\subsection{Berechnung der Wellenlänge}
Die Messung wird von $x_1 = (6 \pm 0,01)\,\unit{\milli\metre}$ bis $x_2 = (11 \pm 0,01)\,\unit{\milli\metre}$ durchgeführt. Außerdem besitzt die Mikrometerschraube ein Hebeluntersetzungsverhältnis von 
$1 : 5,046$.  Demnach ergibt sich für die gemessene Distanz $x = (0,09909\pm 0,0020) \,\unit{\milli\metre}$. Die gemessenen Impulszählraten über diese Distanz sowie die daraus gemittelte Impulszählrate
sind in der Tabelle \ref{tab:Impulszählrate} aufgelistet.
\begin{table}[H]
    \centering
    \caption{Gemesse Impulszählraten über eine Distanz von $x = (0,09909\pm 0,0020) \,\unit{\milli\metre}$ zur Bestimmung der Wellenlänge des Diodenlasers.}
    \label{tab:Impulszählrate}
    \begin{tblr}{colspec={c}}
        \toprule
        $z$\\
        \midrule
        3000\\
        3004\\
        3001\\
        2986\\
        2997\\
        2854\\
        3006\\
        3008\\
        3070\\
        3057\\
        \midrule
        $\overline{z} = 3000\pm50$\\
        \bottomrule
    \end{tblr}
\end{table}
Anhand von $\overline{z}$, der Distanz $x$ und der Gleichung (/ref{eqn:}) wird die Wellenlänge des Diodenlasers bestimmt. Deraus ergibt sich
$$\lambda _{\text{exp}} = (661 \pm 12)\,\unit{\nano\metre}\,.$$
Die Wellenlänge des Diodenlasers laut Herstellerangabe lautet
$$\lambda = 645\,\unit{\nano\metre}\,.$$

\subsection{Berechnung des Brechungsindex von Luft}
Für Messwerte zur Bestimmung der Brechungsindices und die gemittelten Werte sind in der Tabelle {\ref{tab:Vakuum}} aufgeführt. Bei dieser Messung wird
eine Gaszelle mit einer Länge von $D = 50 \,\unit{\milli\metre}$ verwendet. 
\begin{table}[H]
    \centering
    \caption{Gemesse Impulszählraten bei der Evakuierung von der Gaszelle zur Bestimmung der Brechungsindices.}
    \label{tab:Vakuum}
    \begin{tblr}{colspec={c c}}
        \toprule
        $z_{\text{Evak.}}$ & $z_{\text{Belüft.}}$\\
        \midrule
        10&  22\\
        13&  29\\
        9 &  29\\
        8 &  29\\
        9 &  26\\
        \midrule
        $\overline{z}_{\text{Evak.}} = 9,8\pm1,7$ & $\overline{z}_{\text{Belüft.}} = 27,0\pm2,8$\\
        \bottomrule
    \end{tblr}
\end{table}
Die Brechungsindices werden mithilfe der Gleichung (\ref{eqn:}) berechnet. Hierbei ist $T_0 = 0\,\unit{\celsius} = 273,15\,\unit{\kelvin}$ und $\rho_0 = 1013,25\,\unit{\hecto\pascal}$
bei Normalbedingung verwendet. Außerdem ist die Umgebungstemperatur $T=21,6\,\unit{\celsius} = 294,75\,\unit{\kelvin}$ und der Druck durch die Pumpe lautet
$\Delta \rho = 700 \,\unit{\milli\metre}\ce{Hg} = 933,10\,\unit{\hecto\pascal}$. Das benötigte $\Delta n$ wird durch die Gleichung (\ref{eqn:}) und der Wellenlänge $\lambda$ laut Herstellerangabe 
bestimmt. Demnach ergeben sich
\begin{align*}
    \Delta n_{\text{Evak.}} &= (6,3\pm1,1)\cdot 10^{-5}\\
    \Delta n_{\text{Belüft.}} &= (17,4\pm1,8)\cdot 10^{-5}\,.
\end{align*}
Daraus folgt für die Brechungsindices
\begin{align*}
    n_{\text{Evak.}} &=(1,000074\pm0,000013)\\
    n_{\text{Belüft.}} &= (1,000204\pm0,000021)\,.
\end{align*}
