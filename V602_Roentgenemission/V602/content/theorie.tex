\section{Zielsetzung}
\label{sec:Zielsetzung}

\section{Theorie}
\label{sec:Theorie}

\subsection{Vorbereitungsaufgaben}
\label{sec:Vorbereitungsaufgaben}
Zur Vorbereitung sollen die Energien der Kupfer Alpha Linie und der Kupfer Beta Linie recherchiert werden. Zu diesen Energien wird der Glanzwinkel 
Theta des Briggs Kristalls mit Formel (\ref{}) bestimmt. Der Briggs Kristall ist ein LiFI Kristall mit Gitterkonstante $d = 201,4 \unit[\pico\meter]$.
Die sich ergebenden Werte sind 
\begin{align}
    K_\alpha = 8 \unit[\kilo\electronvolt]\\
    \theta_\alpha = 22,63 \unit[\degree] \\
    K_\beta = 8,91 \unit[\kilo\electronvolt]\\
    \theta_\beta = 20,21 \unit[\degree] .
\end{align}


\begin{table}[H]
    \centering
    \caption{Gemessene fünffache Schwingungsdauer bei einer Länge von $32,5\, \unit{\centi\meter}$.}
    \label{tab:EinzelSchwingung_L1}
    \begin{tblr}{colspec={c c}}
        \toprule
        linkes Pendel & rechtes Pendel\\ 
        $5\, T_{\text{l}, 1}\,\left[\unit{\second}\right]$ & $5\, T_{\text{r}, 1}\,\left[\unit{\second}\right]$  \\
        \midrule
        6,21 & 6,19 \\
        6,07 & 6,24 \\
        6,15 & 6,24 \\
        6,19 & 6,30 \\
        6,24 & 6,05 \\
        6,10 & 6,15 \\
        6,22 & 6,18 \\
        6,14 & 6,18 \\
        6,19 & 6,14 \\
        6,35 & 6,25 \\
        \bottomrule
    \end{tblr}
  \end{table}