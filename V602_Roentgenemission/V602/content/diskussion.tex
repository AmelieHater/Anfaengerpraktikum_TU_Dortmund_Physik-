% rel. Abweichungen
% ----------------
% Braggwinkel 0% Abweichung
% ----------------
% sigma_2 Abweichung: 0.677766238385163%
% sigma_3 Abweichung: 8.120530761168153%
% ----------------
% Theta_K_Sr Abweichung: 0.18165304268846116%
% Theta_K_Zr Abweichung: 0.7113821138211411%
% Theta_K_Br Abweichung: 0.5303030303030191%
% Theta_K_Zn Abweichung: 6.666666666666658%
% Theta_K_Ga Abweichung: 6.960845245494101%
% ----------------
% E_Sr Abweichung: 0.16203200173880325 %
% E_Zr Abweichumg: 0.6930393655753725 %
% E_Br Abweichung: 0.5200594346563124 %
% E_Zn Abweichung: 6.018036972081926 %
% E_Ga Abweichung: 6.361406883763256 %
% ----------------
% sigma_Sr Abweichung: 8.703657326179295%
% sigma_Zr Abweichung: 12.897563560205421%
% sigma_Br Abweichung: 7.008758569233188%
% sigma_Zn Abweichung: 31.12267590414942%
% sigma_Ga Abweichung: 7.429171778249866%
% ----------------
% R_inf Abweichung: 5.418122331685387%
\section{Diskussion}
\label{sec:Diskussion}
Die Abweichungen des Braggwinkels, der Abschirmkonstante der Kupferröntgenröhre, des Braggwinkels der verschiedenen Absorptionsmaterialien und der 
Absorptionsenergien sind geringfügig. Die Messung des Braggwinkels ist besonders genau, die Abweichung beträgt $0 \,\%$. Die Abweichung der Abschirmkonstante $\sigma_2$ 
beträgt $0,68 \,\%$, die der Abschirmkonstante $\sigma_3$ beträgt $8,12 \,\%$. 
Die Abweichung des Braggwinkels beträgt bei Strontium $0,18 \,\%$, bei Zirconium $0,71 \,\%$, bei Brom $0,53 \,\%$, bei Zink $6,67 \,\%$ und bei Gallium $6,96 \,\%$. 
Auffällig ist der plötzliche Anstieg des Fehlers bei den letzten beiden Proben. Dies setzt sich bei den folgenden Größen weiter fort. Die Abweichung der
Absorptionsenergie bei Strontium ist $0,16 \,\%$, bei Zirconium $0,69 \,\%$, bei Brom $0,52 \,\%$, bei Zink $6,01 \,\%$ und bei Gallium $6,36 \,\%$. Für die Abschirmkonstanten sehen
die Abweichungen wie folgt aus: Bei Strontium  $8,70 \,\%$, bei Zirconium $12,90 \,\%$, bei Brom $7,01 \,\%$, bei Zink $31,12 \,\%$ und bei Gallium $7,43 \,\%$.
Die Abweichung der Rydberkonstante liegt bei $5,42 \,\%$. \\ 
Vor allem die Abweichungen bei Zink und Gallium fallen höher als die anderen Abweichungen aus. Dies könnte darin begründet liegen, dass die ersten drei Proben von 
uns aufgenommen wurden, Zink und Gallium jedoch von unseren Gruppenpartnern. Dies lässt nur eine bedingte Vergleichbarkeit zu, da wir keine Kontrolle über die 
Einstellungen des Programms hatten. 
Vor allen Dingen bei Zink war der theoretische Braggwinkel nicht im Messbereich enthalten, wodurch große Abweichungen entstehen. Die sonstigen kleinen Abweichungen 
lassen sich dadurch erklären, dass nicht händisch experimentiert werden musste und somit wenig Spielraum für Fehler gegeben war. 