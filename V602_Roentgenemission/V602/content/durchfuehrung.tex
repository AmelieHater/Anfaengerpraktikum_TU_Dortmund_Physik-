\section{Durchführung}
\label{sec:Durchführung}
Der gesamte Versuch lässt sich in verschiedene Abschnitte gliedern. Zuerst wird die Braggbedingung überprüft, dann das Emissionsspektrum einer Kupferröntgenröhre 
gemessen und als letztes das Absorptionsspektrum verschiedener Metalle aufgenommen. 
\subsection{Bragg Bedingung}
Um die Bragg Bedingung zu überprüfen wird im Computerprogramm der Kristallwinkel fest auf $14 \unit{\degree}$ gestellt. Desweitern wird eingestellt, 
dass das Geiger-Müller-Zählrohr in einem Winkelbereich von $26 \unit{\degree}$ bis $30 \unit{\degree}$ misst bei einem Winkelzuwachs von $0,1 \unit{\degree}$. Die 
Integrationszeit pro Winkel soll dabei $5 \unit{\second}$ betragen. 

\subsection{Emissionsspektrum einer Kupferröntgenröhre}
Für die Messung der Emissionsspektrum der Kupferröntgenröhre wird das Programm 2:1 Kopplungsmodus angewählt und im Winkelbereich von $4 \unit{\degree}$ bis $26 \unit{\degree}$
in $0,2 \unit{\degree}$ Schritten abgemessen. Die Integrationszeit beträgt wieder $5 \unit{\second}$ pro Winkel. 

\subsection{Absorptionsspektren verschiedener Materialien}
Vor Begin dieser Messung wird ein Absorber vor das Geiger-Müller-Zählrohr geschraubt. In diesem Versuch werden Absorber aus Strontium, Brom, Zirconium, Zink und Gallium
verwendet. Das Absorptionsspektrum jedes einzelnen Absorbers wird in $0,1 \unit{\degree}$ Schritten gemessen. Die Meßzeit pro Winkel beträgt $30 \unit{\second}$. 
Für Strontium wird im Winkelbereich von $10 \unit{\degree}$ bis $12 \unit{\degree}$ gemessen, für Brom im Bereich von $12 \unit{\degree}$ bis $14 \unit{\degree}$, 
für Zirconium im Bereich von $9 \unit{\degree}$ bis $11 \unit{\degree}$, für Zink im Bereich $19 \unit{\degree}$ bis $21 \unit{\degree}$ und für Gallium im 
Bereich von $17 \unit{\degree}$ bis $19 \unit{\degree}$. 
