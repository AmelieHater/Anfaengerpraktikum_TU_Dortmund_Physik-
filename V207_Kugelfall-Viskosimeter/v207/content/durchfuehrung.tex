\section{Durchführung}
Zuerst wird die Dichte zweier Glaskugel mithilfe von Gleichung (\ref{eqn:DichtefunktionKugel}) bestimmt. 
Dazu wird der Durchmesser gemessen und das Volumen daraus bestimmt. Mithilfe der durch den Übungsleiter 
bekannte Masse wird danach die Dichte berechnet. 
Anschließend wird der Fuß des Viskosimeters mithilfe der Libelle waagerecht ausgerichtet. 
Danach wird das Viskosimeter mit destilliertem Wasser gefüllt und die Kugel hineingegeben. 
Dabei ist darauf zu achten, dass sich keine kleinen Luftblasen am Rand oder an der Kugel absetzen. 
Falls diese Auftreten, sollten sie mithilfe eines Glasstabes vorsichtig entfernt werden. 
\label{sec:Durchführung}