\section{Durchführung}
\label{sec:Durchführung}
Zuerst wird der Durchmesser zweier Glaskugeln gemessen und das Volumen der Kugeln mithilfe von Gleichung
(\ref{eqn:VolumenKugel}) berechnet. Dieses Volumen und die durch den Übungsleiter gegebene Masse 
wird verwendet, um die Dichte der Kugeln mit Gleichtung 
(\ref{eqn:DichtefunktionKugel}) zu bestimmen.  \\
Anschließend wird der Fuß des Viskosimeters mithilfe der Libelle waagerecht ausgerichtet. 
Danach wird das Viskosimeter mit destilliertem Wasser gefüllt und die Kugel hineingegeben. 
Dabei ist darauf zu achten, dass sich keine kleinen Luftblasen am Rand oder an der Kugel absetzen. 
Falls diese Auftreten, sollten sie mithilfe eines Glasstabes vorsichtig entfernt werden. \\Danach wird 
das Viskosimeter mithilfe des Stopfens verschlossen. Zu Beginn der Messung der Fallzeit wird eine Seite 
des Viskosimeters als oben definiert, damit im Folgenden ein Unterschied gemacht werden kann zwischen 
"in Richtung der oberen Seite fallen" (auch "hoch" genannt) und "nicht in die Richtung der oberen Seite fallen"
(auch "runter" genannt). Dies wird neben der Fallzeit vermerkt. Die Zeit wird mit einer Stoppuhr gemessen. 
Es muss darauf geachtet werden, dass
die Kugel ihre konstante Endgeschwindigkeit erreicht hat, bevor sie in den Messbereich eintritt.\\ 
Die Fallzeit der kleinen Kugel wird insgesamt 20 Mal gemessen, 10 Mal "hoch"\,\,und 10 Mal "runter", auf einer Strecke von 
$10$ \unit{\centi\meter} zu fallen. Danach wird die Zeit der größeren Kugel bestimmt, 
die sie braucht um $5$ \unit{\centi\meter} zu fallen.
Es werden insgesamt 10 Messwerte erhoben (5 Mal "hoch"\,\,und 5 Mal "runter").\\
Danach wird die Apparatekonstante $K$ der großen Kugel mit Gleichung (\ref{eqn:KFunktion}) berechnet. Dazu wird die gegebene 
Apparatekonstante für die kleine Kugel $K_{kl}$ %= 0.07640 \unit[per-mode=fraction]{\milli\pascal\per\cubic\centi\meter}$ 
verwendet. \\
Anschließend werden durch Verwendung der großen Kugel Messdaten zu verschiedenen Temperaturen des 
destillierten Wassers aufgenommen. 
Es werden 10 unterschiedliche Temperaturen verwendet, die auf einer Skala von $\SI{20}{\celsius}$ bis 
$\SI{55}{\celsius}$ liegen. Bei jeder einzelnen Temperatur werden 4 Messwerte aufgenommen (jeweils 2 Mal 
"hoch"\,\,und "runter").
Zwischen Erhöhen der Wassertemperatur und dem Aufnehmen der Messwerte muss eine möglichst konstante Zeit 
gewartet werden, damit sich die Temperatur des destillierten Wassers im inneren Zylinder der Temperatur des umliegenden 
Wassers angleicht. \\
Mithilfe der Messungen wird im Anschluss die dynamische Viskosität $\eta(T)$ des destillierten Wassers und die 
Reynoldszahl bestimmt. Zusätzlich wird überprüft, ob die Strömung laminar ist.

