\section{Zielsetzung}
\label{sec:Zielsetzung}
Das Ziel des Versuches ist die Temperaturabhängigkeit der dynamischen 
Viskosität von destilliertem Wasser zu bestimmen. Dazu wird das 
Kugelfallviskosimeter nach Höppler verwendet. Außerdem wird die Reynoldszahl 
berechnet und benutzt, um herauszufinden ob es sich bei der Strömung um 
laminare oder turbulente handelt. 
%
%
%
\section{Theorie}
    \label{sec:Theorie}
    Bewegt sich ein Körper durch ein Medium hindurch, wirkt eine Reibungskraft 
    $\vec{F}_{R}$, die unter anderem von der Berührungsfläche und der Geschwindigkeit
    des Körpers abhängt. Je nach Strömungsart kann diese Kraft 
    unterschiedliche Abhängigkeiten haben, bei dem Kugelfallviskosimeter nach 
    Höppler ist von einer laminaren Strömung auszugehen. 
    Dies wird in der Auswertung durch die
    Berechnung der Reynoldszahl überprüft. Eine experimentspeziefische 
    Reynoldszahl über ca. 2300 weißt auf eine turbulente Strömung hin, eine die
    darunter liegt auf eine laminare Strömung. Die Reynoldszahl berechnet sich
    über 
    \begin{equation}
        Re = \frac{\rho_{\text{M}} \cdot \bar{v} \cdot d}{\eta}\,. 
        \label{eqn:Reynoldszahl}
    \end{equation}
    Dabei bezeichnet $\rho_{\text{M}}$ die Dichte des Mediums, $\bar{v}$
    die mittlere Geschwindigkeit des Körpers, $d$ die eine charakteristische 
    Länge (beim Kugelviskosimeter ist dies der Durchmesser der Röhre)
    und $\eta$ die dynamische Viskosität des Mediums. \\
    Die Reibungskraft ist bei laminarer Strömung die Stokessche Reibung
    \begin{equation}
        \vec{F}_{R} = 6 \cdot \pi \cdot \eta \cdot v \cdot r \, , 
        \label{eqn:Stokesreibungskraft}
    \end{equation}
    hier bereits an die Symmetrie einer Kugel mit Berührungsfläche 
    $A = 6 \cdot \pi \cdot r$ angepasst, wobei $r$ der Radius der Kugel ist, 
    $\eta$ ist die dynamische Viskosität des Mediums, eine Materialkonstante,
    $v$ ist die Fallgeschwindigkeit des Körpers. \\
%
%
%
        \subsection{Kugelfallviskosimeter nach Höppler}
            Die beschriebene Theorie ist die Grundlage der Funktionalität des Viskosimeters 
            nach Höppler. Es besteht aus einem geschlossenen Glaszylinder, 
            welcher mit einer leichten Neigung am Fuß befestigt
            ist. Dieser Zylinder ist um 180° drehbar. 
            Innerhalb des Zylinders ist Wasser, welches durch Schläuche mit ein Termostat verbunden ist,
            welches das Wasser aufheizen 
            kann.
            Durch den Zylinder führt eine Glasröhre, die von außen durch Stöpsel 
            verschlossen werden kann. Auf der Glasröhre sind 3 Striche, die jeweils 
            einen Abstand von 5 \unit{\centi\meter} haben. In die innere Röhre kann
            ein Medium und eine Kugel eingefüllt werden. Diese können durch das umliegende Wasser mit dem 
            Termostat erwärmt werden. Dadurch werden Wirbel im inneren Medium vermieden. 
            Bei diesem Experiment hat die größere verwendete Kugel näherungsweise den Durchmesser der Röhre. 
            Die leichte Neigung der Apperatur wurde gewählt, um die unkontrollierte Bewegung zu vermeiden, 
            die bei einer senkrecht herabfallen Kugel entstehen würde. Auf die herabfallende Kugel
            wirken während des Falls drei Kräfte: Die Gravitationskraft $\vec{F}_{G} = m \cdot \vec{g}$, 
            die die Kugel nach unten beschleunigt, die Auftriebskraft $\vec{F}_{A}$ und die 
            Reibungskraft $\vec{F}_{R}$. Die Reibungskraft und Auftriebskraft wirken entgegengesetzt zur Schwerkraft. 
            Aufgrund der Kräfte beschleunigt
            die Kugel im Medium zuerst bis sie eine konstante Endgeschwindigkeit erreicht, wenn sich das Kräftegleichgewicht 
            $\vec{F}_{G} = \vec{F}_{A} + \vec{F}_{R}$ eingestellt hat.
            Die Viskosität $\eta$ kann durch diese empirische Formel  
            \begin{align}
                \eta &= K \cdot \left( \rho_{K}\,-\,\rho_{M} \right) \cdot t \label{equ:EmpirischeEtaFunktion} \\
                \Leftrightarrow K &= \frac{\eta}{ \left( \rho_{K}\,-\,\rho_{M} \right) \cdot t }
                \label{eqn:KFunktion}
            \end{align} 
            beschrieben werden. $K$ ist dabei eine Proportionalitätskonstante, $\rho_{K}$ die Dichte der Kugel und $\rho_{M}$
            die Dichte des Mediums. Die Dichte der Kugel kann durch die Formel 
            \begin{equation}
                \rho_{K} = \frac{m_{K}}{V_{K}}
                \label{eqn:DichtefunktionKugel}
            \end{equation}
            bestimmt werden, wo bei $m_{K}$ die Masse der Kugel ist und $V_{K}$ das Volumen der Kugel.
            Das Volumen kann aus dem Durchmesser $d_{K}$ durch
            \begin{equation}
                V_{K} = \frac{4}{3} \cdot \pi \cdot \left( \frac{d_{K}}{2} \right)^3
                \label{eqn:VolumenKugel}
            \end{equation} 
            berechnet werden. 
    %
    %
    %
        \subsection{Vorbereitungsaufgaben}
            \label{sec:Vorbereitungsaufgaben}
            \textbf{Wann bezeichnet man eine Strömung als "laminar"?}\\
            \\
            Eine Strömung ist dann laminar, wenn die einzelnen benachbarten Schichten 
            des Mediums ohne sich gegenseitige Störung aneinander vorbeibewegen und 
            keine Wirbel entstehen. \\ 
            \\
            %
            \textbf{Wie lautet die Dichte und die dynamische Viskosität von 
            destilliertem Wasser als Funktion der Temperatur?}\\
            \\
            Die Dichte von destilliertem Wasser kann unterhalb von  
            $\SI{100}{\celsius}$ nicht als temperaturabhängige Formel 
            beschrieben werden.
            Die Dichte $\rho_{\text{Wasser}}$ bei $\SI{20}{\celsius}$ beträgt $998.207$ 
            \unit[per-mode=fraction]{\kilo\gram\per\meter\tothe{3}}.(Quelle ist https://studyflix.de/chemie/dichte-wasser-1574)\\
            Außerdem gibt es auch keine spezielle Funktion für die dynamische
            Viskosität von destilliertem Wasser, die Andradesche Gleichung 
            \begin{equation}
            \eta (T) = A \cdot e^{\frac{B}{T}}
            \label{eqn:AndradescheGleichung}
            \end{equation} 
            gilt auch für destilliertes Wasser. $A$ und $B$ sind Konstanten und $T$ ist die Temperatur in Kelvin. 
%\cite{sample}