\section{Zielsetzung}
\label{sec:Zielsetzung}
Das Ziel des Versuches ist die Temperaturabhängigkeit der dynamischen 
Viskosität von destilliertem Wasser zu bestimmen. Dazu wird das 
Kugelfallviskosimeter nach Höppler verwendet. Außerdem wird die Reynoldszahl 
berechnet und benutzt, um herauszufinden ob es sich bei der Strömung um 
laminare oder turbulente handelt. 
%
%
%
\section{Theorie}
\label{sec:Theorie}
Bewegt sich ein Körper durch ein Medium hindurch, wirkt eine Reibungskraft 
$\vec{F}$, die unter anderem von der Berührungsfläche und der Geschwindigkeit
des Körpers abhängt. Je nach Strömungsart kann diese Kraft 
unterschiedliche Abhängigkeiten haben, bei dem Kugelfallviskosimeter nach 
Höppler ist von einer laminaren Strömung auszugehen. 
Dies wird in der Auswertung durch die
Berechnung der Reynoldszahl überprüft. Eine experimentspeziefische 
Reynoldszahl über ca. 2300 weißt auf eine turbulente Strömung hin, eine die
darunter liegt auf eine laminare Strömung. Die Reynoldszahl berechnet sich
über 
\begin{equation}
    Re = \frac{\rho_{\text{M}} \cdot \bar{v} \cdot d}{\eta}\,. 
    \label{equ:Reynoldszahl}
\end{equation}
Dabei bezeichnet $\rho_{\text{M}}$ die Dichte des Mediums, $\bar{v}$
die mittlere Geschwindigkeit des Körpers, $d$ die eine charakteristische 
Länge und $\eta$ die dynamische Viskosität des Mediums. \\
Die Reibungskraft ist bei laminarer Strömung die Stokessche Reibung
\begin{equation}
    F_{R} = 6 \cdot \pi \cdot \eta \cdot v \cdot r \, , 
    \label{equ:Stokesreibungskraft}
\end{equation}
hier bereits an die Symmetrie einer Kugel mit Berührungsfläche 
$A = 6 \cdot \pi \cdot r$ angepasst, wobei $r$ der Radius der Kugel ist, 
$\eta$ ist die dynamische Viskosität des Mediums, eine Materialkonstante,
$v$ ist die Fallgeschwindigkeit des Körpers. \\
%
%
%
\subsection{Kugelfallviskosimeter nach Höppler}
Diese Theorie ist die Grundlage der Funktionalität des Viskosimeters 
nach Höppler. Dieses besteht aus einem geschlossenen Glaszylinder, 
welcher mit einer leichten Neigung am Fuß befestigt
ist. Dieser Zylinder ist um 180 \degree drehbar. Innerhalb des 
Zylinders ist Wasser und ein Termostat, welches das Wasser aufheizen kann.
Durch den Zylinder durch führt eine Glasröhre, die von außen durch Stöpsel 
verschlossen werden kann. Auf der Glasröhre  
%
%
%
\section{Vorbereitungsaufgaben}
\label{sec:Vorbereitungsaufgaben}
\textbf{Wann bezeichnet man eine Strömung als "laminar"?}\\
\\
Eine Strömung ist dann laminar, wenn die einzelnen benachbarten Schichten 
des Mediums ohne sich gegenseitige Störung aneinander vorbeibewegen und 
keine Wirbel entstehen. \\ 
\\
%
\textbf{Wie lautet die Dichte und die dynamische Viskosität von 
destilliertem Wasser als Funktion der Temperatur?}\\
\\
Die Dichte von destilliertem Wasser kann unterhalb von  
$\SI{100}{\celsius}$ nicht als temperaturabhängige Formel 
beschrieben werden.
Die Dichte bei $\SI{20}{\celsius}$ beträgt $998.207$ 
\unit[per-mode=fraction]{\kilo\gram\per\meter\tothe{3}}.(Quelle ist https://studyflix.de/chemie/dichte-wasser-1574)\\
Außerdem gibt es auch keine spezielle Funktion für die dynamische
Viskosität von destilliertem Wasser, die Andradesche Gleichung 
\begin{equation}
 \eta (T) = A \cdot e^{\frac{B}{T}}
 \label{equ:AndradescheGleichung}
\end{equation} 
gilt auch für destilliertes Wasser. $A$ und $B$ sind Konstanten und $T$ ist die Temperatur in Kelvin. 
%\cite{sample}