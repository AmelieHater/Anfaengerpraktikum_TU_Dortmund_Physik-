\section{Zielsetzung}
\label{sec:Zielsetzung}
Das Ziel des Versuches ist die Temperaturabhängigkeit der dynamischen 
Viskosität von destilliertem Wasser zu bestimmen. Dazu wird das Kugelfall-
Viskosimeter nach Höppler verwendet. Außerdem wird die Reynoldszahl 
berechnet und benutzt, um herauszufinden ob es sich bei der Strömung um 
laminare oder turbulente handelt. 
\section{Theorie}
\label{sec:Theorie}
Bewegt sich ein Körper durch ein Medium hindurch, wirkt eine Reibungskraft 
$/vv(F)$, die unter anderem von der Berührungsfläche und der Geschwindigkeit
des Körpers abhöngt. Je nach Strömungsart kann diese Kraft 
unterschiedliche Abhängigkeiten haben, bei dem Kugelfallviskosimeter nach 
Höppler ist von einer laminaren Strömung auszugehen. 
Dies wird in der Auswertung durch die
Berechnung der Reynoldszahl überprüft. Eine experimentspeziefische 
Reynoldszahl über ca. 2300 weißt auf eine turbulente Strömung hin, eine die
darunter liegt auf eine laminare Strömung. Die Reynoldszahl berechnet sich
über $$Re = \frac{\rho_{\text{Fl}}\,v\,d}{\eta}.$$ \\
Die Reibungskraft ist bei laminarer Strömung die Stokessche Reibung, hier 
bereits an die Symmetrie einer Kugel mit Berührungsfläche $6 \pi r$
angepasst, wobei $r$ der Radius der Kugel ist. 
$$F_{R} = 6\,\pi\,\eta\,v\,r.$$ \\
$\eta$ ist die dynamische Viskosität des Mediums, eine Materialkonstante.
$v$ ist die Fallgeschwindigkeit des Körpers und 

\section{Vorbereitungsaufgaben}
\label{sec:Vorbereitungsaufgaben}
\textbf{Wann bezeichnet man eine Strömung als "laminar"?}\\
Eine Strömung ist dann laminar, wenn die einzelnen benachbarten Schichten 
des Mediums ohne sich gegenseitige Störung aneinander vorbeibewegen und 
keine Wirbel entstehen. 
%
\textbf{Wie lautet die Dichte und die dynamische Viskosität von 
destilliertem Wasser als Funktion der Temperatur?}\\
Die Dichte von destilliertem Wasser kann unterhalb von 100 
$\units{\celsius}$ nicht als temperaturabhängige Formel beschrieben werden.
Die Dichte bei 20 $\units{\celsius}$ beträgt $$998.207 
\frac{\units{kilogramm}}{\units{\meter}^3}$.
\cite{sample}