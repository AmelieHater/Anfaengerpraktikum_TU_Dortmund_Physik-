% Masse kl: 4.4531 # in g (gegeben)
% Masse gr: 4.9528 # in g (gegeben)
% Durchmesser kl. Kugel: 1.5570+/-0.0010 (in cm)
% Durchmesser gr. Kugel: 1.5760+/-0.0010 (in cm)
% Volumen der kl. Kugel: 1.976+/-0.004 (in cm^3)
% Volumen der gr. Kugel: 2.050+/-0.004 (in cm^3)
% Dichte der kl. Kugel: 2.253+/-0.004 (in g/cm^3)
% Dichte der gr. Kugel: 2.416+/-0.005 (in g/cm^3)
% Gemittelte Fallzeit (hoch) kl: 12.20+/-0.13, gr: 34.75+/-0.16
% Gemittelte Fallzeit (runter) kl: 12.14+/-0.11, gr: 34.73+/-0.09
% Viskosität hoch: 1.170+/-0.013 (in mPa*s)
% Viskosität runter: 1.164+/-0.011 (in mPa*s)
% Apparatekonstante K_gr_h: 0.02374+/-0.00030 (in mPa*cm^3/g)
% Apparatekonstante K_gr_r: 0.02364+/-0.00025 (in mPa*cm^3/g)
% Reynoldsche Zahl Re_kl_h: 110.2+/-2.4
% Reynoldsche Zahl Re_kl_r: 111.3+/-2.0
% Reynoldsche Zahl Re_gr_h: 19.35+/-0.24
% Reynoldsche Zahl Re_gr_r: 19.46+/-0.19
%
% K_kl = 0.07640 (in m*Pa*cm^3/g) (gegeben)
% dichte_wasser = 0.998207 (in g/cm^3) (Internet)
% aus Plot:
% ln(A) = -5.567 ± 0.097
% B = 1680+/-30
% A = 0.0038+/-0.0004
%
\section{Auswertung}
\label{sec:Auswertung}
\subsection{Viskosität von Wasser bei Raumtemperatur}
\label{sec:Viskosität von Wasser}
Zunächst wird die Dichte der kleinen Glaskugel $\rho_{\text{kl}}$ durch die Gleichung 
(\ref{eqn:DichtefunktionKugel}) und Gleichung (\ref{eqn:VolumenKugel}) bestimmt. 
Dafür werden die gegebene Masse $m_{\text{kl}} = 4.4531\,\unit{\gram}$ und der 
gemessene Durchmesser $d_{\text{kl}}= \left(1.5570 \pm 0.0010\right)\,$ \unit{\centi \meter} 
verwendet.
$$\rho_{\text{kl}} = \left(2.253 \pm 0.004\right)\;\unit[per-mode=fraction]{\gram\per\cubic\centi\meter}$$ 
Die betrachteten Fallzeiten der kleinen Kugel sind in dieser Tabelle erfasst:
\begin{table}[H]
  \centering
  \caption{Gemessene Fallzeiten der kleinen Kugel bei einer Strecke von $10\, \unit{\centi\meter}$}
  \begin{tblr}{colspec={c c}}
      \toprule
      $t_{\text{Runter}}\, \left[\unit{\second}\right]$ & $t_{\text{Hoch}}\, \left[\unit{\second}\right]$ \\ 
      \midrule
      12.32 & 12.20\\
      12.18 & 12.35\\
      12.15 & 12.43\\
      12.24 & 12.23\\
      12.18 & 12.19\\
      12.18 & 12.26\\
      12.17 & 12.17\\
      11.92 & 12.10\\
      12.01 & 11.91\\
      12.07 & 12.19\\
      \bottomrule
  \end{tblr}
\end{table}
\noindent
Anhand dieser Messdaten erhält man die folgenden Zeiten, die für die Berechnung der Viskosität benötigt werden:
\begin{align*}
  t_{\text{kl,r}} &= \left(12.14\pm0.11\right) \, \unit{\second}\\
  t_{\text{kl,h}} &= \left(12.20\pm0.13\right) \, \unit{\second}
\end{align*}
Die Viskosität von Wasser bei Raumtemperatur lässt sich mithilfe der Gleichung (\ref{eqn:EmpirischeEtaFunktion}) 
und der angegebenen Apparatekonstante der kleinen Glaskugel $K_{kl}= 0.0760 \,\unit[per-mode=fraction,inter-unit-product=\cdot]{\milli\pascal\cubic\centi\meter\per\gram}$
bestimmen.
\begin{align*}
  \eta_{\text{Hoch}} &= \left(1.170\pm0.013 \right)\, \unit[inter-unit-product=\cdot]{\milli\pascal\second}\\
  \eta_{\text{Runter}} &= \left(1.164\pm0.011 \right) \,\unit[inter-unit-product=\cdot]{\milli\pascal\second}
\end{align*}
%
\subsection{Apparatekonstante der großen Glaskugel}
\label{sec:Apparatekonstante der großen Glaskugel}
Vorab wird erneut mit der Gleichung (\ref{eqn:DichtefunktionKugel}) und der Gleichung (\ref{eqn:VolumenKugel})
die Dichte der großen Glaskugel bestimmt. Hierbei beträgt die gegebene Masse $m_{\text{gr}}= 4.9528 \;\unit{\gram}$ und 
der gemessene Durchmesser $d_{\text{gr}}=\left(1.5760\pm0.0010\right)\, \unit{\centi\meter}$.
$$\rho_{\text{gr}} = \left(2.416\pm0.005\right)\;\unit[per-mode=fraction]{\gram\per\cubic\centi\meter}$$ 
%
Die betrachteten Fallzeiten der großen Kugel sind in dieser Tabelle erfasst:

\begin{table}[H]
  \centering
  \caption{Gemessene Fallzeiten der großen Kugel bei einer Strecke von $5\,\unit{\centi\meter}$}
  \begin{tblr}{colspec={c c}}
      \toprule
      $t_{\text{Runter}}\, \left[\unit{\second}\right]$ & $t_{\text{Hoch}}\, \left[\unit{\second}\right]$ \\ 
      \midrule
      34.61 & 34.70 \\
      34.78 & 34.64 \\
      34.69 & 35.00 \\
      34.87 & 34.86 \\
      34.69 & 34.56 \\
      \bottomrule
  \end{tblr}
\end{table}
Aus den Messdaten werden die folgenden Zeiten bestimmt:
%
\begin{align*}
  t_{\text{gr,r}} &= \left( 34.73\pm0.09\right) \, \unit{\second}\\
  t_{\text{gr,h}} &= \left(34.75\pm0.16 \right) \, \unit{\second}
\end{align*}
Folglich wird mit der Gleichung (\ref{eqn:KFunktion}) und den zuvor berechneten Viskositäten die 
Apparatekonstante der großen Glaskugel bestimmt.
\begin{align*}
  K_{\text{gr,r}} &= \left(0.02364\pm0.00025  \right) \, \unit[per-mode=fraction,inter-unit-product=\cdot]{\milli\pascal\cubic\centi\meter\per\gram}\\
  K_{\text{gr,h}} &= \left(0.02374\pm0.00030  \right) \, \unit[per-mode=fraction,inter-unit-product=\cdot]{\milli\pascal\cubic\centi\meter\per\gram}
\end{align*}
%
\subsection{Bestimmung der Reynoldschen Zahl}
Durch einsetzen der Werte in die Gleichung (\ref{eqn:Reynoldszahl}) ergeben sich für die 
kleine Glaskugel diese Reynoldszahlen:
\begin{align*}
  Re_{\text{kl,r}} &= \left(111.3\pm2.0\right)\\
  Re_{\text{kl,h}} &= \left(110.2\pm2.4\right)
\end{align*}
Analog erfogt die Berechnung der Reynoldszahlen für die große Glaskugel.
\begin{align*}
  Re_{\text{gr,r}} &= \left(19.46\pm0.19\right)\\
  Re_{\text{gr,h}} &= \left(19.35\pm0.24\right)
\end{align*}
Aus den Reynoldszahlen erschließt sich, dass sowohl bei der kleinen als auch bei der großen Glaskugeln
eine laminare Strömung entsteht, da alle Reynoldzahlen deutlich kleiner sind als $Re=2300$.
%
\subsection{Temperaturabhängigkeit der Viskosität von Wasser}
Für die Auswertung der Temperaturabhängigkeit der Viskosität von Wasser wird der natürliche
Logarithmus von der Andradeschen Gleichung (\ref{eqn:AndradescheGleichung}) gebildet. Somit ensteht ein
linearer Zusammenhang zwischen dem Kehrwert von der Temperatur und dem natürlichen Logarithmus
von der dynamischen Viskosität $\eta$.
\begin{align}
  \begin{split}
   \eta (T) &= A \cdot e^{\frac{B}{T}}\\
    \Rightarrow \ln\left(\eta\left(T\right)\right) &= \ln \left(B\right) \cdot \frac{1}{T} + \ln\left(A\right)
    \label{eqn:lnEta}
  \end{split}
\end{align}
Mithilfe von Gleichung (\ref{eqn:EmpirischeEtaFunktion}) 
Die mithilfe von Gleichung (\ref{eqn:EmpirischeEtaFunktion}) gemessenen Fallzeiten der großen Kugel bei variierter Temperatur sind in der folgenden Tabelle
aufgeführt:
\begin{table}[H]
  \centering
  \caption{Gemessene Fallzeiten der großen Kugel bei unterschiedlichen Temperaturen}
  \begin{tblr}{colspec={c c c c c}}
      \toprule
      $T\, \left[\unit{\celsius}\right]$ & $t_{\text{h1}}\, \left[\unit{\second} \right]$ & $t_{\text{r1}}\, \left[\unit{\second} \right]$ & $t_{\text{h2}}\, \left[\unit{\second} \right]$ & $t_{\text{r2}}\, \left[\unit{\second} \right]$ & $\rho(T)\,\left[\unit[per-mode=fraction]{\gram\per\cubic\centi\meter} \right]$ (Q\cite{dichte})\\
      \midrule
      31\pm1 & 28.76 & 28.26 & 28.27 & 28.80 & 0.99534 \\
      33\pm1 & 27.16 & 27.03 & 27.12 & 27.07 & 0.99470 \\
      34\pm1 & 26.76 & 27.01 & 26.75 & 26.76 & 0.99437 \\ 
      36\pm1 & 26.16 & 26.06 & 26.16 & 26.24 & 0.99369 \\
      40\pm1 & 24.32 & 24.23 & 24.40 & 24.24 & 0.99222 \\
      42\pm1 & 23.52 & 23.46 & 23.36 & 23.36 & 0.99144 \\
      43\pm1 & 22.84 & 22.81 & 22.87 & 23.27 & 0.99104 \\
      44\pm1 & 22.38 & 22.64 & 22.23 & 21.80 & 0.99063 \\
      47\pm1 & 21.32 & 21.69 & 21.38 & 21.24 & 0.98936 \\
      49\pm1 & 20.64 & 20.81 & 20.58 & 20.41 & 0.98848 \\
      \bottomrule
  \end{tblr}
\end{table}
Im Abschnitt \ref{sec:Viskosität von Wasser} wurde bereits festgestellt, dass die dynamischen Viskositäten 
der beiden Fallrichtungen bei Raumtemperatur keine signifikante Differenz aufweisen. Aus diesem Grund werden
bei dieser Auswertung die Fallrichtungen \glqq hoch\grqq\,und \glqq runter\grqq\,nicht mehr getrennt voneinander
betrachtet. Demnach erhält man für die jeweiligen Temperaturen die folgenden Fallzeiten:
\begin{table}[H]
  \centering 
  \caption{Fallzeiten der großen Kugel bei unterschiedlichen Temperaturen}
  \begin{tblr}{colspec={c c c c}}
      \toprule
      $T\, \left[\unit{\celsius}\right]$ & $t\, \left[\unit{\second} \right]$ & $\rho(T)\,\left[\unit[per-mode=fraction]{\gram\per\cubic\centi\meter} \right]$ (Q\cite{dichte}) & $\eta\left(T\right)\, \left[ \unit[inter-unit-product=\cdot]{\milli\pascal\second}\right] $\\
      \midrule
      31\pm1 & 28.510 \pm 0.250 & 0.99534 & 0.962 \pm 0.014 \\
      33\pm1 & 27.095 \pm 0.065 & 0.99470 & 0.908 \pm 0.011 \\
      34\pm1 & 26.885 \pm 0.125 & 0.99437 & 0.907 \pm 0.012 \\  
      36\pm1 & 26.110 \pm 0.050 & 0.99369 & 0.882 \pm 0.011 \\
      40\pm1 & 24.275 \pm 0.045 & 0.99222 & 0.821 \pm 0.010 \\
      42\pm1 & 23.490 \pm 0.030 & 0.99144 & 0.795 \pm 0.010 \\
      43\pm1 & 22.825 \pm 0.015 & 0.99104 & 0.772 \pm 0.009 \\
      44\pm1 & 22.510 \pm 0.130 & 0.99063 & 0.762 \pm 0.010 \\
      47\pm1 & 21.505 \pm 0.185 & 0.98936 & 0.729 \pm 0.011 \\
      49\pm1 & 20.725 \pm 0.085 & 0.98848 & 0.703 \pm 0.009 \\
      \bottomrule
  \end{tblr}
\end{table}
Die in der Tabelle aufgelisteten dynamischen Viskositäten $\eta\left(T\right)$ sind mit der Gleichung (\ref{eqn:EmpirischeEtaFunktion}) 
und den temperaturabhängigen Dichten berechnet.
Mithilfe dieser Daten und der Gleichung (\ref{eqn:lnEta}) lassen sich die Messwerte sowie
deren Unsicherheiten graphisch darstellen. 
\begin{figure}[H]
  \centering
  \includegraphics[width=0.92\textwidth]{plot.pdf}
  \caption{Lineare Ausgleichsgerade der Andradeschen Gleichung}
  \label{fig:plot}
\end{figure}
Anschließend erfolgt eine lineare Regression aus 
der Andradeschen Gleichung (\ref{eqn:AndradescheGleichung}). 
Anhand dieser linearen Regression und der Gleichung (\ref{eqn:lnEta}) werden die Konstanten $A$ und $B$ 
ermittelt. Hierbei entspricht $A$ dem Ordinatenabschnitt und $B$ der Steigung der 
linearen Ausgleichsgeraden.
\begin{align*}
  \ln\left(A\right) &= -5.567 \pm 0.097 \Longleftrightarrow A = 0.0038\pm0.0004\\
  B &= 1680\pm30
\end{align*}
%Siehe \autoref{fig:plot}!