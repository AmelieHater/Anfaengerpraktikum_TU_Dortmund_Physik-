\section{Durchführung}
\label{sec:Durchführung}
Zuerst wird der Versuchsaufbau so justiert, dass das Lichtspektrum der Hg-Lampe scharf auf der Photozelle abgebildet wird. 
Danach wird der Photostrom $I$ in Abhängigkeit von der Spannung des Gegenfeldes der Photozelle $U$ gemessen innerhalb eines Bereiches von $-2$ bis 
$19 \, \unit{\volt}$. Für die ersten Messwerte wird eine geringe Schrittweite von $0,2 \, \unit{\volt}$ verwendet, für die folgenden Messwerte
eine Schrittweite von $3 \, \unit{\volt}$. Diese Messung wird bei einer bestimmten Wellenlänge des Spektrums durchgeführt. 
Nach der Messung wird der Spalt neu eingestellt, sodass die halbe Intensität im Vergleich zur ersten Messreihe erreicht wird. 
Die bei diesen Einstellungen vorgenommene Messreihe hat eine Schrittweite von $3 \, \unit{\volt}$ im gesamten Messbereich. 
Danach wird eine neue Spektrallinie vermessen. Bei dieser Messung und bei allen folgenden werden ungefähr 10 Messwerte aufgenommen 
mit einer Schrittweite von $0,2 \, \unit{\volt}$ beginnend bei der Spannung für die der korrespondierende Photostrom $0$ ist. 
Diese Messung wird für $3$ unterschiedliche Spektrallinien durchgeführt. 