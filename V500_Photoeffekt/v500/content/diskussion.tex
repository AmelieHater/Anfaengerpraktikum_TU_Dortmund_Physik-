% h_theo = 4,135e-15
\section{Diskussion}
\label{sec:Diskussion}
Für die Grenzspannungen sind keine Theoriewerte vorhanden. Allerdings liegen die abgelesenen Grenzspannungen
im Fehlerbereich der berechneten Grenzspannungen durch die Ausgleichsgeraden. Jedoch fällt beim Vergleichen der 
Planck-Konstante und der Austrittsarbeit auf, dass die berechneten Grenzspannungen bessere Werte erzielen.
Für die Planck-Konstante ergibt sich eine relative Abweichung von $28,7\,\%$ für die Methode mit den abgelesenen Grenzspannungen.
Dahingegen ergibt sich mit den berechneten Grenzspannungen eine relative Abweichung von $22,2\,\%$ zur Planck-Konstante. \\Laut der Anleitung \cite{anleitungV500}
besteht die Kathode aus einer oxidierten Silberschicht, auf der Kalium aufgebracht ist. Daher werden die Austrittsarbeiten mit den Theoriwerten
$\Phi_{\text{A, Ag}} = 4,05 - 4,60$ und $\Phi_{\text{A, Ka}} = 2,25$ verglichen Q\cite{Austrittsarbeit}. Mithilfe der gemessenen Grenzspannungen lautet die Abweichung
im Vergleich zu Eisen $74,6\,\% - 77,6\,\%$ und im Vergleich zu Kalium $54,3\,\%$. Für die berechneten Werte sind die relativen Abweichung
$71,9\,\% - 75,2\,\%$ von Eisen und $49,3\,\%$ von Kalium. Diese Abweichungen könnten daran liegen, dass die Metallschicht der Kathode
verunreinigt sein könnte. Die relativen Abweichungen können während des Experimentierens dadurch zustande gekommen sein, dass die Fokussierung
des Spektrums sehr unsicher war. Zudem war die violette, zuletzt gemessene Linie des Spektrums sehr schwach und hatte daher eine höheren Fehlerbereich. 
