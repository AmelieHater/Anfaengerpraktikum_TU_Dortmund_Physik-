\section{Diskussion}
\label{sec:Diskussion}
%Abweichung zu 3/2:  16.0
%Abweichung der Kathodentemperatur:  29.640227027534127
%Abweichung der Austrittsarbeit:  13.43612334801763
Bei den fünf gemessenen Kennlinien stellen sich bei allen der 
Sättigungsstrom ein. Daher sollte dieser mit hoher Genauigkeit ablesbar sein. 
Für den Exponenten der Langmuir-Schottkysch-Gleichung ergibt sich eine relative Abweichung von 
$16 \, \%$. Diese Abweichung ist gering genug, als dass die Gleichung als bestätigt angesehen werden kann.
Die relative Abweichung zwischen der experimentell bestimmten Kathodentemperatur von $1500 \, \unit{\kelvin}$ 
und der aus der Leistungsbilanz errechneter Wert von $2131,90 \, \unit{\kelvin}$ beträgt $29,64 \, \%$.
Die Austrittsarbeit kann mit dem Theoriewert $\bar{\Phi}_{Theo} = 4,54 \, \unit{\eV}$ vergleichen werden. 
Dadurch ergibt sich eine relative Abweichung von $13,44 \, \%$. 
Diese Abweichung können dadurch zustande gekommen sein, dass die Anzeige des Messgeräts beim Ablesen der Stromstärke 
teils stark geschwankt hat. Außerdem war das Einstellen der verschiedenen Spannungen teils 
problematisch dadurch, dass durch den Versuchsaufbau hindurch auf das Gerät geschaut werden musste. 
Trotzdessen sind die Abweichungen verhältnismäßig gering. 