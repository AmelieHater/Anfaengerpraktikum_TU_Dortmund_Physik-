% Diskussion:
% rel. Abweichung tau: 8.688650054783055 %
% rel. Abweichung Steigung s: 1 %
\section{Diskussion}
\label{sec:Diskussion}
Die Abweichungen werden mithilfe der Formel
\begin{equation*}
    \text{rel. Abweichung} = \frac{|\text{exp. Wert} - \text{theo. Wert}|}{\text{theo. Wert}}
  \end{equation*}
berechnet. 
Die relative Abweichung zwischen der abgelesenen Totzeit am Oszilloskop und der berechneten Totzeit lautet $8,69\,\%$. 
Bei der Plateausteigung beträgt die Abweichung zwischen dem experimentellen und dem theoretischen Wert $1,00\,\%$. 
Diese kleinen Abweichugen lassen sich dadurch erklären, dass der Aufbau nur von dem Praktikumsleiter durchgeführt wurde und daher 
die Fehlerquellen minimiert wurden. Außerdem hängt der theoretisch bestimme Wert der Steigung von dem experimentell bestimmten Arbeitspunkt
ab. Dies führt dazu, dass die Werte nicht unabhängig voneinander betrachtet werden können.
