\section{Zielsetzung}
\label{sec:Zielsetzung}
Das Ziel des Versuchs ist die Beschäftigung mit der Charakteristik eines Geier-Müller 
Zähler. Dazu wird die Kennlinie der Stoffes \ce{^{204}Tl} analysiert und 
die Totzeit des verwendeten Geier-Müller Zählrohrs bestimmt.
\section{Theorie}
\label{sec:Theorie}
\subsection{Aufbau und Funktion des Zählrohrs}


\subsection{Kennlinie}

\subsection{Vorbereitungsaufgaben}
\label{sec:Vorbereitungsaufgaben}
Zur Vorbereitung wird die Halbwertszeit und die Zerfallskanäle von \ce{^{204}Tl}
recherchiert. Die Halbwertszeit beträgt $3,783$ Jahre und \ce{^{204}Tl}
zerfällt zu $2,92 \, \%$ durch einen $\beta⁺$ in \ce{^{204}Hg} und zu $97,08 \, \%$ 
durch einen $\beta⁻$ Zerfall in \ce{^{204}Pb} \cite{vorbereitung}. 
Außerdem sollte die Zährate $N \geq 10.000$ sein, um eine statistische Messunsicherheit
von $1 \, \%$ zu erhalten, da die statistische Messunsicherheit proportional zu 
$\sqrt{N^{-1}}$.
