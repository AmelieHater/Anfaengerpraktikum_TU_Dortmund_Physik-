\section{Diskussion}
\label{sec:Diskussion}
Die relative Abweichung der Experimentalwerte von 
den Theoriewerten wird durch die Formel 
$$\Delta x = \left|\frac{x_{exp} - x_{theo}}{x_{theo}} \right|$$
bestimmt. 
Der Literaturwert der Anregungsenergie von Quecksilber 
ist $\Delta E_{theo} = 4,9 \, \unit{\eV}$. Daraus ergibt 
sich eine Beschleunigungsspannungsdifferenz zwischen Maxima 
$\Delta U_B = 4,9 \, \unit{\volt}$. Die relative Abweichung 
vom Experimentalwert mit $U_B = 5,1 \unit{\volt}$ ist $4,08 \, \%$. 
Die Abweichung kann dadurch zustande gekommen sein, dass die Achse des X-Y-Schreiber 
nicht linear ist, was zur Zeit der Durchführung des Experiments nicht bekannt war. 
Daher ist es überraschend, dass die Abweichung nicht höher ausfällt. Zusätzlich ist das Ablesen 
vom Millimeterpapier ebenfalls mit Unsicherheiten verknüpft. 
Dies macht die gesamte Auswertung unsicherer. Außerdem war es nicht 
möglich bei einer konstanten Temperatur aufzuzeichnen, wenn die Temperatur 
höher war als die Außentemperatur. Während der Messung sank die Temperatur 
teils um mehr als 10 °C. 