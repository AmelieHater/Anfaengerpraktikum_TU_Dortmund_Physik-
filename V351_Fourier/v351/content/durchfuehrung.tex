\section{Durchführung}
\label{sec:Durchführung}
In diesem Versuch werden mit einem Scanner, Oberwellengenerator, Funktions-Erzeuger und Oszilloskop eine 
Fourier-Synthese und Analyse für verschiedene Funktionen durchgeführt.
\subsection{Fourier-Synthese verschiedener Funktionen}
Für die Fourier-Synthese werden die in der Vorbereitungsaufgabe berechneten 
$\frac{1}{n}$- bzw $\frac{1}{n^2}$-Abhängigkeiten der Amplituden verwendet. 
Bei den $\frac{1}{n}$-abhängigen Funktionen lassen sich mit dem Oberwellengenerator
die $n$-ten Oberwellen für $n \in \{1,9\}$ einstellen. Dafür wird die Grundfrequenz 
($n=1$) durch $n$ dividiert, um die benötigte Amplitude zu erhalten. Diese Amplitude 
wird dann für das jeweilige $n$ mit dem Oberwellengenerator eingestellt und lässt sich am
Scanner ablesen.
Anschließend werden die jeweiligen Phasen der Oberwellen am Oberwellengenerator angepasst. 
Hierfür wird ein digitales Oszilloskop zur visuellen Darstellung verwendet. 
Das Gleiche wird erneut für die $\frac{1}{n^2}$-abhängige Funktion durchgeführt. Allerdings
wird für die $n$-te Amplitude die Grundfrequenz durch $n^2$ dividiert. 
\subsection{Fourier-Analyse verschiedener Funktionen}
Für die Fourier-Analyse werden mit dem Funktions-Erzeuger jeweils eine Dreiecks-, Rechteck- und 
Sägezahnschwingung erzeugt. Zunächst wird am Oszilloskop in der $x,t$ Einstellung überprüft,
ob die jeweilige Schwingung übermittelt wird. Dann wird der Mathe-Modus und die \glqq{Fast Fourier Transformation}\grqq{}(FFT)
im Oszilloskop eingestellt. In der $x,y$ Einstellung ist dann das Linienspektrum der Funktionen zu erkennen. Hier werden die jeweiligen Peaks mit
dem integrierten Courser im Oszilloskop abgelesen und notiert.