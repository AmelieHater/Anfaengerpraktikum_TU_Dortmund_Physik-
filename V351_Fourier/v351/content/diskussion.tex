\section{Diskussion}
\label{sec:Diskussion}
Die relative Abweichung zwischen dem theoretischen und dem experimentellen Wert wird bestimmt durch
$$\text{rel. Abweichung} = \frac{|\text{exp. Wert} - \text{theo. Wert}|}{\text{theo. Wert}}\,.$$
Bemerkenswert ist, dass alle gemessenen Werte sehr nah an den erwarteten Theoriewerten liegen. 
Für die Sägezahnspannung wurde $- 1,0036 \pm 0,0035$ für $\symup{d}_1$ gemessen, was einer relativen Abweichung von $- 3,6 \cdot 10^{-3}$ entspricht, bei einem Theoriewert 
von $- 1$. Das für die Dreieckspannung berechnete $\symup{d}_2$ beträgt $- 1,9809 \pm 0,0072$. Dies führt bei einem Theoriewert von $- 2$ zu einer relativen Abweichung von 
$- 9,55 \cdot 10^{-3}$. Der Theoriewert der Rechteckspannung ist $\symup{d}_{3,\text{theo}} = - 1$. Die relative Abweichung bei einem gemessen Wert von 
$\symup{d}_3 = - 0,9939 \pm 0,0022$ ist daher $- 6,1 \cdot 10^{-3}$. \\
Diese äußerst geringen Fehler können dadurch erklärt werden, dass das Ablesen der Messwerte digital mithilfe eines Cursers erfolgte, wodurch die Unsicherheit, 
die durch das Ablesen von Menschen entsteht, minimiert wurde. Außerdem musste bei diesem Versuch wenig von den messenden Personen ohne digitale Anzeige eingestellt
werden. Daher waren alle Einstellungen und Messungen sehr genau. Die höchste Messunsicherheit, die bei der Dreieckspannung entstand, kann durch 
im Vergleich zu den anderen Messreihen wenig aufgenommene Messdaten erklärt werden. Dadurch, dass die Peaks mit $n^{-2}$ abfiehlen, war nach wenigen Messdaten 
die Höhe der Peaks so gering, dass sie im Bereich des Hintergundrauschens waren und die Werte damit nicht mehr verlässlich abgelesen werden konnten.  
