\section{Diskussion}
\label{sec:Diskussion}
Die relative Abweichung zwischen dem theoretischen und dem experimentellen Wert wird bestimmt durch
$$\text{rel. Abweichung} = \frac{|\text{exp. Wert} - \text{theo. Wert}|}{\text{theo. Wert}}\,.$$
Bemerkenswert ist, dass alle gemessenen Werte sehr nah an den erwarteten Theoriewerten liegen. 
Für die Sägezahnspannung wurde $- 1,0036 \pm 0,0035$ für $\symup{d}_1$ bestimmt, was bei einem Theoriewert 
von $- 1$ einer relativen Abweichung von $-0,36\,\%$ entspricht. Das für die Dreieckspannung berechnete $\symup{d}_2$ beträgt $- 1,9809 \pm 0,0072$. Dies führt bei einem Theoriewert von $- 2$ zu einer relativen Abweichung von 
$- 0,955\,\%$. Der Theoriewert der Rechteckspannung ist $\symup{d}_{3,\text{theo}} = - 1$. Die relative Abweichung bei einem gemessen Wert von 
$\symup{d}_3 = - 0,9939 \pm 0,0022$ ist daher $-0,61\,\%$. \\
Diese äußerst geringen Fehler können dadurch erklärt werden, dass das Ablesen der Messwerte digital mithilfe eines Coursers erfolgt, wodurch die Unsicherheit, 
die durch das Ablesen von Menschen entsteht, minimiert wird. 
Außerdem werden bei diesem Versuch viele Einstellungen mit einer digitalen Anzeige eingestellt. Daher sind alle Einstellungen und Messungen sehr genau. Die höchste Messunsicherheit, 
die bei der Dreieckspannung entseht, kann durch die, im Vergleich zu den anderen Messreihen, wenig aufgenommenen Messdaten erklärt werden. Dadurch, dass die Peaks mit $n^{-2}$ abfallen, 
ist nach wenigen Messdaten die Höhe der Peaks so gering, dass diese im Bereich des Hintergundrauschens sind und die Werte damit nicht mehr verlässlich abgelesen werden können.  
