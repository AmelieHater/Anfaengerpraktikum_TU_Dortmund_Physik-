\section{Auswertung}
\label{sec:Auswertung}
  \subsection{Berechnung der Winkelrichtgröße $D$}
  Zur Bestimmung der Winkelrichtgröße $D$ wurde die Kraft $F$ in Abhängigkeit des Auslenkungswinkels $\varphi$ bei festem Abstand von 
  $0,19975 \,\unit{\meter}$ gemessen. Diese Werte sind in Tabelle (\ref{tab:F_von_phi}) zu sehen. Die Winkelrichtgröße $D$ wird durch Gleichung 
  (\ref{eqn:Winkelrichtgröße}) bestimmt und ist ebenfalls in Tabelle (\ref{tab:F_von_phi}) eingetragen.
  Das gemittelte Ergebnis ist 
  $\bar{D} = (21,0 \pm 0,8) \cdot 10^{-3} \,\unit{\newton\meter}$.
  
  \begin{table}[H]
    \centering 
    \caption{Kraft in Abhängigkeit vom Auslenkungswinkel}
    \label{tab:F_von_phi}
    \begin{tblr}{colspec={c c c}}
        \toprule
        $\varphi [^{\circ}]$ & F [\unit{\newton}] & D [\unit{\newton\meter}]\\
        \midrule
        20 & 0,026 & 0,0149\\
        30 & 0,050 & 0,0191\\
        40 & 0,068 & 0,0195\\  
        50 & 0,090 & 0,0206\\
        60 & 0,120 & 0,0229\\
        70 & 0,136 & 0,0222\\
        80 & 0,156 & 0,0223\\
        90 & 0,184 & 0,0234\\
        100 & 0,20 & 0,0229\\
        110 & 0,21 & 0,0218\\
        \bottomrule
    \end{tblr}
  \end{table}
  

  \subsection{Berechnung des Eigenträgheitmoments}
  Um das Eigenträgheitsmoment der Drillachse $I_{\text{Drill}}$ zu bestimmen, wurde die fünffache Periodendauer $T$ in Abhängigkeit vom Abstand $a$
  bei einer Auslenkung von $90 ^{\circ}$ gemessen. Diese Messwerte sind in Tabelle (\ref{tab:Bestimmung_I_D}) vermerkt. 
  \begin{table}[H]
    \centering 
    \caption{Fünffache Periodendauer in Abhängigkeit vom Abstand}
    \label{tab:Bestimmung_I_D}
    \begin{tblr}{colspec={c c}}
        \toprule
        $a \,[\unit{\meter}]$ & $5 \cdot T \,[\unit{\second}]$ \\
        \midrule
        0,050 & 14,40 \\
        0,075 & 16,57 \\
        0,100 & 18,60 \\
        0,125 & 21,41 \\
        0,150 & 30,10 \\
        0,175 & 26,78 \\
        0,200 & 29,94 \\
        0,225 & 32,75 \\
        0,250 & 36,60 \\
        0,300 & 42,50 \\
        \bottomrule
    \end{tblr}
  \end{table}
  $I_{\text{Drill}}$ wird durch die Verbindung $$I_{\text{gemessen}} = I_{\text{Drill}} + 2 \cdot I_{\text{Zh, verschoben}}$$ berechnet. 
  Dabei ist $I_{\text{Zh, verschoben}}$ nach Satz von Steiner: 
  $$I_{\text{Zh, verschoben}} = I_{\text{Zh}} + m \cdot a^2$$ 
  Mithilfe der Gleichung (\ref{eqn:TragheitmomentAusSchwingungsdauer})
  können $I_{\text{Drill}}$ und $T^2$ wie folgt ausgedrückt werden:
  \begin{align}
    I_{\text{Drill}} &= \frac{T^{2} \cdot D}{\left(2 \pi\right)^{2}} - 2 \cdot \left(m \left(\frac{r^{2}}{4} + \frac{h^{2}}{12} \right) + m \cdot a^2 \right) \\
    \Leftrightarrow T^2 &= \frac{8 \pi^2 \cdot m}{D} \cdot a^2 + \frac{4 \pi^2  \cdot m \cdot I_{\text{Drill}}}{D} + \frac{8\pi^2 \cdot m}{D} \cdot \left( \frac{r^2}{4} + \frac{h^2}{12} \right)
  \end{align}
  In Graph (\ref{fig:plot}) $T^2$ wird gegen $a^2$ aufgetragen und durch lineare Regression der Form $y = m_{\text{Steigung}} \cdot x + n$ werden $m$ und $n$ bestimmt.
  $m$ beträgt $713,1192 s^2/m^2$ und $n$ beträgt $8.564 s^2$. 
  \begin{figure}[H]
    \centering
    \includegraphics{plot.pdf}
    \caption{Auftragung von Periodenzeit $T^2$ gegen Abstand $a^2$}
    \label{fig:plot}
  \end{figure}
  Mithilfe dieser Regression und der obrigen Gleichung lässt sich das Eigenträgheitsmoment $I_{\text{Drill}}$ wie folgt bestimmen: 
  \begin{equation}
    I_{\text{Drill}} = \frac{D \cdot n}{4 \pi^2} - \frac{1}{2} \cdot m \cdot r^2 - \frac{1}{6} \cdot m \cdot h^2
  \end{equation}
  $r$ bezeichnet dabei den Radius und $h$ die Höhe der zylindischen Gewichte, die für die Messung verwendet wurden. Die Messungen dieser beiden
  Größen werden in Tabelle (\ref{tab:Durchmesser_Höhe_Blaues}) und (\ref{tab:Durchmesser_Höhe_Rotes}) aufgeführt. Für die Rechnung wird er Mittelwert beider Gewichte 
   $r = (22,550 \pm 0,010) \cdot 10^{-3}\unit{\meter}$ und $h = (20,334 \pm 0,008) \cdot 10^{-3} \unit{\meter}$ verwendet.
  \begin{table}[H]
    \centering 
    \caption{Messungen der Höhe und des Durchmessers des blauen Gewichts}
    \label{tab:Durchmesser_Höhe_Blaues}
    \begin{tblr}{colspec={c c}}
        \toprule
        Höhe blaues Gewicht $\,[\unit{\meter}]$ & Durchmesser blaues Gewicht$\,[\unit{\meter}]$ \\
        \midrule
        0,02034 & 0,0450 \\
        0,02030 & 0,0451 \\
        0,02030 & 0,0451 \\
        0,02032 & 0,0451 \\
        0,02030 & 0,0451 \\
        \bottomrule
    \end{tblr}
  \end{table}

  \begin{table}[H]
    \centering 
    \caption{Messungen der Höhe und des Durchmessers des roten Gewichts}
    \label{tab:Durchmesser_Höhe_Rotes}
    \begin{tblr}{colspec={c c}}
        \toprule
        Höhe rotes Gewicht $\,[\unit{\meter}]$ & Durchmesser rotes Gewicht$\,[\unit{\meter}]$ \\
        \midrule
        0,02038 & 0,04506 \\
        0,02036 & 0,04510 \\
        0,02032 & 0,04510 \\
        0,02036 & 0,04518 \\
        0,02036 & 0,04516 \\
        \bottomrule
    \end{tblr}
  \end{table}
  Dies ergibt ein Eigenträgheitmoment $I_{\text{Drill}}$ von $(4,46 \pm 0,18) \cdot 10^{-3} \unit{\kilo\gram\meter\squared}$.

  \subsection{Eigenträgheitmoment der Kugel}
    \subsubsection{Theoretischer Wert}
    Das Eigenträgheitmoment der Kugel $I_{\text{K}}$ kann durch Gleichung (\ref{eqn:TragheitKugel}) berechnet werden. 
    $r = \frac{d}{2}$ ist dabei der aus dem Mittelwert der gemessenen Durchmesser berechnete Wert $r = 0.14661 \pm 0.00020 \unit{\meter}$. 
    Die gemessenen Durchmesser sind in Tabelle(\ref{tab:Durchmesser_Kugel}) dargestellt. $m$ ist $1,1742 \unit{\kilo\gram}$. 
    
    \begin{table}[H]
      \centering 
      \caption{Gemessene Durchmesser der Kugel}
      \label{tab:Durchmesser_Kugel}
      \begin{tblr}{colspec={c}}
          \toprule
          Durchmesser Kugel $\,[\unit{\meter}]$ \\
          \midrule
          0,14660 \\
          0,14590 \\
          0,14660 \\
          0,14685 \\
          0,14710 \\
          \bottomrule
      \end{tblr}
    \end{table}
    Damit beträgt der Theoriewert des Trägheitsmoment der Kugel $(2,524 \pm 0,007) \cdot 10^{-3} \unit{\kilo\gram\meter\squared}$.

    \subsubsection{Gemessener Wert}
    Die für die Rechnung benötigte Periodendauer der Kugel ist in Tabelle(\ref{tab:T5_Kugel}) dargestellt. 
    Sie wurde bei fester Auslenkung um 90° gemessen.
    \begin{table}[H]
      \centering 
      \caption{Gemessene fünfache Periodendauer der Kugel}
      \label{tab:T5_Kugel}
      \begin{tblr}{colspec={c}}
          \toprule
          Periodendauer $\,[\unit{\second}]$ \\
          \midrule
          9,53 \\
          9,50 \\
          9,35 \\
          9,72 \\
          9,40 \\
          9,53 \\
          9,38 \\
          9,31 \\
          9,32 \\
          9,44 \\
          \bottomrule
      \end{tblr}
    \end{table}
    Der Mittelwert einer Periodendauer $T_{\text{K}} = (1,890 \pm 0,008) \unit{\second}$ wurde für die Berechnung des 
    Trägheitsmomentes $I_{\text{K, gemessen}}$ mithilfe von Gleichung (\ref{eqn:TragheitmomentAusSchwingungsdauer}) verwendet. 
    $I_{\text{K, gemessen}}$ ist $(1,90 \pm 0,08) \cdot 10^{-3} \unit{\kilo\gram\meter\squared}$.

  \subsection{Eigenträgheitmoment der Schiebe}
    \subsubsection{Theoretischer Wert}
    Das Eigenträgheitmoment der Scheibe kann durch Gleichung (\ref{eqn:TragheitZylinder}) berechnet werden. $m$ beträgt $0,4237 \unit{\kilo\gram}$.
    Das zur Berechnung verwendete $r$ wurde aus dem Mittelwert der Durchmessermessungen berechnet. 
    Die Durchmessermessungen sind in Tabelle (\ref{tab:Durchmesser_Scheibe}) vermerkt. Das mithilfe dieser Werte berechnete, theoretische 
    Trägheitsmoment der Scheibe ist $I_{\text{S}} = (2,552 \pm 0,006) \cdot 10^{-3} \unit{\kilo\gram\meter\squared}$. 
    \begin{table}[H]
      \centering 
      \caption{Gemessene Durchmesser der Scheibe}
      \label{tab:Durchmesser_Scheibe}
      \begin{tblr}{colspec={c}}
          \toprule
          Durchmesser Scheibe $\,[\unit{\meter}]$ \\
          \midrule
          0,22010 \\
          0,21890 \\
          0,22010 \\
          0,21920 \\
          0,21915 \\
          \bottomrule
      \end{tblr}
    \end{table}
    \subsubsection{Gemessener Wert}
    Der gemessene Wert des Trägheitmoment der Scheibe $I_{\text{S, gemessen}}$ wird mithilfe der Gleichung (\ref{eqn:TragheitmomentAusSchwingungsdauer}) berechnet.
    $T$ ist dabei der Mittelwert der einfachen Periodendauer. Die gemessene fünffache Periodendauer ist in Tabelle (\ref{tab:T5_Scheibe}) dargestellt.
    \begin{table}[H]
      \centering 
      \caption{Gemessene fünfache Periodendauer der Scheibe}
      \label{tab:T5_Scheibe}
      \begin{tblr}{colspec={c}}
          \toprule
          Periodendauer $\,[\unit{\second}]$ \\
          \midrule
          9,28 \\
          9,22 \\
          9,31 \\
          9,34 \\
          9,37 \\
          9,31 \\
          9,56 \\
          9,31 \\
          9,44 \\
          9,31 \\
          \bottomrule
      \end{tblr}
    \end{table}
    Für $I_{\text{S, gemessen}}$ wird folgender Wert berechnet: 
    $(1,85 \pm 0,07) \cdot 10^{-3} \unit{\kilo\gram\meter\squared}$ 
  
  \subsection{Trägheitsmoment der Puppe}
    \subsubsection{1. Position der Puppe}
      \textbf{Theoretischer Wert}\\
        Zur Berechnung des Trägheitsmoments der Puppe in Position 1 werden eine Reihe von Vereinfachungen angenommen. Der Kopf wird genauso
        wie die Arme, Beine und der Torso als zylindrisch angenommen und die Dichte als homogen. Der Durchmesser der einzelnen Zylinder wird 
        als Mittelwert der Messungen angenommen. Diese Messungen sind in Tabelle (\ref{tab:Durchmesser_Puppe}) aufgeführt. 

        \begin{table}[H]
          \centering 
          \caption{Durchmesser der Puppenteile}
          \label{tab:Durchmesser_Puppe}
          \begin{tblr}{colspec={c c}}
              \toprule
              Kopfdurchmesser $\,[\unit{\meter}]$ & Armdurchmesser $\,[\unit{\meter}]$ & Torsodurchmesser $\,[\unit{\meter}]$ & Beindurchmesser $\,[\unit{\meter}]$\\
              \midrule 
              0,02726 & 0,00824 & 0,02330 & 0,01900 \\
              0,02786 & 0,01480 & 0,02708 & 0,01734 \\
              0,01510 & 0,01200 & 0,03874 & 0,01178 \\
              0,01650 & 0,01412 & 0,03230 & 0,01514 \\
              0,02300 & 0,01520 & 0,03834 & 0,01694 \\
              & 0,01138 & 0,04230 & 0,01304 \\
              & 0,01270 & 0,04020 & 0,00924 \\
              & 0,01020 & 0,03600 & 0,01708 \\
              & 0,01180 & 0,03472 & 0,01620 \\
              & 0,01384 & 0,03800 & 0,01344 \\
              \bottomrule
          \end{tblr}
        \end{table}
        Mithilfe der gemittelten Durchmesser werden die Volumina der einzelnen Körperteile berechnet, um dadurch die Masse der Einzelteile zu bestimmen. 
        Bei dieser Berechnung wird angenommen, dass die Dichte der Puppe homogen ist. Die durch diese Formel 
        $$V = \pi \cdot r^2 \cdot h$$
        berechneten Volumen der Puppenteile, sowie die daraus anteilig am Gesamtgewicht von $0,1696 \,\unit{\kilo\gram}$ bestimmte Massen sind in Tabelle (\ref{tab:Volumen_Puppe}) 
        und (\ref{Massen_Puppe})gelistet. 
        \begin{table}[H]
          \centering 
          \caption{Volumina der Puppenteile}
          \label{tab:Volumen_Puppe}
          \begin{tblr}{colspec={c c c c}}
              \toprule
              Kopf $\,[10^{-5}\,\unit{\meter\tothe{3}}]$ & Arm $\,[10^{-5}\,\unit{\meter\tothe{3}}]$ & Torso $\,[10^{-5}\,\unit{\meter\tothe{3}}]$ & Bein $\,[10^{-5}\,\unit{\meter\tothe{3}}]$\\
              \midrule 
              0,017 \pm \,\,0,004 & 1,66 \pm \,\,0,18 & 9,50 \pm \,\,1,0 & 3,0 \pm \,\,0,4 \\
              \bottomrule
          \end{tblr}
        \end{table}

        \begin{table}[H]
          \centering 
          \caption{Massen der Puppenteile}
          \label{tab:Massen_Puppe}
          \begin{tblr}{colspec={c c c c}}
              \toprule
              Kopfmasse $\,[\unit{\kilo\gram}]$ & Armmasse $\,[\unit{\kilo\gram}]$ & Torsomasse $\,[\unit{\kilo\gram}]$ & Beinmasse $\,[\unit{\kilo\gram}]$\\
              \midrule 
              0,017 \pm\, 0,004 & 0,0136 \pm\, 0,0015 & 0,078 \pm\, 0,006 & 0,017 \pm\, 0,004 \\
              \bottomrule
          \end{tblr}
        \end{table}
        Mithilfe dieser Massen wird das Trägheitsmoment der einzelnen Puppenteile bestimmt, woraus sich das Gesamtträgheitsmoment der Puppe in 
        Position 1 zusammensetzt. Da alle Gliedmaße als Zylinder genähert wurden, werden die Trägheitsmoment durch Formel (\ref{eqn:TragheitZylinder})
        für den Kopf, den Torso und die Beine und Formel (\ref{eqn:TrägheitZylinderQuer}) für die vom Torso ausgestrecken Arme. 
        Für die Berechnung des Trägheitsmomentes der Beine und Arme wird der Satz von Steiner angewandt. Der dabei verwendete Abstand beträgt bei den 
        Beinen näherungsweise den Radius der Beine und bei den Armen die Summe aus dem Radius des Torsos und der Hälfte der Höhe der Arme. 
        Diese Werte führen zu einem theoretischen Trägheitsmoment der Puppe in Position 1 von $(0,262 \pm 0,025) \cdot 10^{-3}\,\, \unit{\kilo\gram\meter\squared}$.
        \\
        \\
      \textbf{Gemessener Wert} \\
      Die gemessenen fünffachen Periodendauer der Puppe in Position 1 sind in Tabelle (\ref{tab:T5_Positon_1}) vermerkt, unterschieden 
      nach Auslenkung um $90$°
      und $120$°. 
      \begin{table}[H]
        \centering 
        \caption{Fünfache Periodendauer der Puppe in Position 1}
        \label{tab:T5_Positon_1}
        \begin{tblr}{colspec={c c}}
            \toprule
            T bei Auslenkung um 90° $\,[\unit{\second}]$ & T bei Auslenkung um 120° $\,[\unit{\second}]$ \\
            \midrule
            3,10 & 3,15 \\
            3,06 & 3,06 \\
            3,06 & 3,19 \\
            3,03 & 3,00 \\
            3,03 & 3,03 \\
            \bottomrule
        \end{tblr}
      \end{table}
      Da der Mittelwert der Periodendauer bei Auslenkung um $90$° und $120$° annähernd gleich sind, wird davon ausgegangen, dass keine Abhängigkeit der 
      Periodendauer von der Auslenkung besteht und beide Messungen zusammengefasst betrachtet. 
      Der allgemeine Mittelwert der Periodenauer $T$ ist daher $T = 0,614 \pm 0,004 \unit{\second}$. 
      Mithilfe dieses Mittelwertes und Formel (\ref{eqn:TragheitmomentAusSchwingungsdauer}) wird das 
      Trägheitsmoment der Puppe in Position 1 $I_{\text{Pos1,gemessen}}$ berechnet. 
      $I_{\text{Pos1,gemessen}}$ ist $(0,2 \pm 0,008) \cdot 10^{-3}\, \unit{\kilo\gram\meter\squared}$.
            
    \subsubsection{Position 2 der Puppe}
    Position 2 der Puppe ist in Bild (\ref{fig:Position_2}) zu sehen. 
    Es wurde vereinfachend angenommen, dass beide Beine im Winkel von $90$° vom Torso wegzeigen. Die Armposition ist dieselbe wie bei Position 1 der Puppe.
    \begin{figure}[H]
      \centering
      \includegraphics[width=0.5\textwidth]{Position_2.jpg}
      \caption{Puppe in Position 2, seitlich fotographiert.}
      \label{fig:Position_2}
    \end{figure}
      \textbf{Theoretischer Wert} \\
      Für die Berechnung des theoretischen Wertes des Trägheitsmoments der Puppe in Position 2 können die bereits berechneten Trägheitsmomente der
      Arme, des Kopfes und des Torsos übernommen werden, da ihre Position keine Veränderung aufweist im Vergleich zwischen Position 1 und 2. 
      Neu berechnet werden hingegen die Trägheitsmomente der Beine. Für die Berechnung dieser wird Formel (\ref{eqn:TrägheitZylinderQuer}) und
      der Satz von Steiner verwendet. 
      Das theoretisch berechnete Trägheitsmoment der Puppe in Position 2 beträgt demnach $(0,72 \pm 0,05) \cdot 10^{-3}\, \unit{\kilo\gram\meter\squared}$. \\
      \\
      \textbf{Gemessener Wert} \\
      Wie bereits bei Position 1 festgestellt, hängt die Periodendauer nicht von der Auslenkung ab. Daher werden auch bei Position 2 beide Messwertreihen
      zusammen betrachtet. Die gemessene fünfache Periodendauer ist trotz dessen in Tabelle (\ref{tab:T5_Positon_2}) nach Auslenkungswinkel 
      getrennt aufgeführt.
      \begin{table}[H]
        \centering 
        \caption{Fünfache Periodendauer der Puppe in Position 2}
        \label{tab:T5_Positon_2}
        \begin{tblr}{colspec={c c}}
            \toprule
            T bei Auslenkung um 90° $\,[\unit{\second}]$ & T bei Auslenkung um 120° $\,[\unit{\second}]$ \\
            \midrule
            4,72 & 4,85 \\
            4,87 & 4,90 \\
            4,84 & 4,81 \\
            4,69 & 4,87 \\
            4,69 & 4,85 \\
            \bottomrule
        \end{tblr}
      \end{table}
      Der Mittelwert der Periodenauer $T$ ist $T = 0,962 \pm 0,004 \unit{\second}$. 
      Mithilfe dieses Mittelwertes und Formel (\ref{eqn:TragheitmomentAusSchwingungsdauer}) wird das 
      Trägheitsmoment der Puppe in Position 2 $I_{\text{Pos2,gemessen}}$ berechnet. 
      $I_{\text{Pos2,gemessen}}$ ist $ (0,491 \pm 0,020) \cdot 10^{-3}\, \unit{\kilo\gram\meter\squared}$.
        

  
%Siehe \autoref{fig:plot}!