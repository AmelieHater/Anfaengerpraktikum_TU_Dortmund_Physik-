\section{Zielsetzung}
\label{sec:Zielsetzung}
Das Ziel dieses Versuchs ist das Trägheitsmoment von unterschiedlichen rotierenden Objekten zu bestimmen. 
Zudem wird der Satz von Steiner überprüft.
%
\section{Theorie}
\label{sec:Theorie}
Rotationsbewegungen lassen sich mit Hilfe des Drehmoments $M$, des Trägheitsmoment $I$ und 
der Winkelbeschleunigung $\dot{\omega}$ beschreiben. Das Trägheitsmoment einer Punktmasse, 
mit der Masse $m$ und dem Abstand $r$ zu einer Drehachse, lässt wie folgt beschreiben:
\begin{equation}
    I = m \cdot r^{2}\,.
    \label{eqn:TragheitPunktmasse}
\end{equation}
%
Bei einer Rotation eines ausgedehnten Körpers bewegen sich alle Masseelemente mit einer 
Winkelbeschleunigung $\dot{\omega}$ um eine feste Drehachse. Dabei gilt für das Gesamtträgheitsmoment:
\begin{equation}
    I = \sum_i r_{i}^{2} \cdot m_{i}\, .
    \label{eqn:GesamttragheitSumme}
\end{equation}
%
Hier beschreibt $r_{i}$ den Abstand der einzelnen Masselemente $m_{i}$ zur festen Drehachse.
Für infinitisimale Massen gilt für das Gesamtträgheitsmoment:
\begin{equation}
    I = \int r^{2}\,dm \,.
    \label{eqn:GesamttragheitIntegral}
\end{equation}
%
Um bei einem komplexen Körper das Trägheitsmoment zu bestimmen, wird dieser häufig in 
geometrisch einfache Formen aufgeteilt, dessen Trägheitsmomente zu einem Gesamtträgheitsmoment 
addiert werden. Hierbei ist zu beachten, dass die Trägheitsmomente der jeweiligen Formen sich
auf dieselbe Drehachse beziehen. Somit hängt das Drehmoment von der Lage der Drehachse ab. \\
%
Wenn die Drehachse nicht durch den Schwerpunkt des Körpers verläuft und stattdessen parallel 
mit einem Abstand $a$ zu einer Schwerpunktachse verläuft, dann lässt sich mit Hilfe des Satz von Steiner
das Trägheitsmoment berechnen.
\begin{equation}
    I = I_{s} + m \cdot a^{2}
    \label{eqn:SatzVonSteiner}
\end{equation}
%
$I_{s}$ beschreibt das Trägheitsmoment, das sich auf die Schwerpunktachse des Körpers bezieht, und $m$
ist die Masse des Körpers. 
Das Drehmoment ist beschrieben durch:
\begin{align}
    \vec{M} &= \vec{F} \times \vec {r} \text{  bzw.} \label{eqn:DrehmomentVektor} \\
    M &= F \cdot r \cdot \sin \left(\varphi \right) \,, \label{eqn:DrehmomentSkalar}
\end{align}
%
wobei $\vec{F}$ bzw. $F$ die Kraft angibt, die auf den rotierenen Körper wirkt, $\vec{r}$ 
bzw. $r$ der Abstand zur Drehachse ist und der Winkel $\varphi$ beschreibt die Auslenkung zur
Ruhelage. \\
In einem schwingfähigen System wirkt dem Auslenkungswinkel $\varphi$ ein rücktreibendes Drehmoment entgegen.
Ein solches System erzeugt eine harmonische Schwindung mit der Schwingungsdauer: 
\begin{equation}
    T = 2 \pi \sqrt{\frac{I}{D}}\,,
    \label{eqn:Schwingungsdauer}
 \end{equation}
wobei $D$ die Winkelrichgröße und $I$ das Trägheitsmoment ist. 
Das wirkende Drehmoment wird mit Hilfe der Winkelrichtgröße $D$ wie folgt ausgedrückt:
\begin{equation}
   M = D \cdot \varphi\,.
   \label{eqn:DrehmomentMitWinkelrichtgröße}
\end{equation}
%
Die Winkelrichtgröße berechnet sich durch: 
\begin{equation}
   D = \frac{F \cdot r}{\varphi}\,.
    \label{eqn:DrehmomentMitWinkelrichtgröße}
\end{equation}
%
\subsection{Spezifische Trägheitsmomente}
\label{sec:Trägheitsmmente}
Die umgestellte Gleichung (\ref{eqn:Schwingungsdauer}) nach $I$ ergibt:
\begin{equation}
    I = \frac{T^2 \cdot D}{\left(2 \pi\right)^2}\,.
    \label{eqn:DrehmomentMitWinkelrichtgröße}
\end{equation}
%
Das Trägheitmoment einer Kugel mit Masse $m$ und Radius $r$ wird durch: 
\begin{equation}
    I_K = \frac{2}{5}m \cdot r^2 
    \label{eqn:TragheitKugel}
\end{equation}
bestimmt. \\
Das Trägheitsmoment eines Zylinders mit Masse $m$ und Radius $r$ ist gegeben 
durch:
\begin{equation}
    I_Z = \frac{m \cdot r^2}{2}
    \label{eqn:TragheitKugel}
\end{equation}


\subsection{Vorbereitungsaufgaben}
\label{sec:Vorbereitungsaufgaben}
