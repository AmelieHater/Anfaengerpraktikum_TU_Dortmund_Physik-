\section{Auswertung}
\label{sec:Auswertung}

  \subsection{Verdampfungswärme von Wasser bis 1 bar}
  Der zu Anfang gemessene Umgebungsdruck $p_0$ beträgt $985\,\unit{\milli\bar}$ bei einer Umgebungstemperatur von $22 \,\unit{\celsius}$. 
  Die gemessenen Werte für das Druckverhalten bei ansteigendender Temperatur für 
  Druck unter 1 bar ist in Tabelle (\ref{tab:Druck_unter_1_bar}) aufgelistet. 

    \begin{longtblr}[
      caption = {Gemessener Druck $p$ bei verschiedenen Temperaturen $T$},
      label = {tab:Druck_unter_1_bar},
      ]{
      colspec = {c c c c c c},
       }
      \toprule
      $T \, \left[\unit{\celsius}\right]$ & $p \, \left[\unit{\milli\bar}\right]$ & $T \, \left[\unit{\celsius}\right]$ & $p \, \left[\unit{\milli\bar}\right]$ & $T \, \left[\unit{\celsius}\right]$ & $p \, \left[\unit{\milli\bar}\right]$\\ 
    \midrule
      25 \pm 1 & 95 \pm 1   & 51 \pm 1 & 197 \pm 1 & 77 \pm 1 & 438 \pm 1 \\                     
      26 \pm 1 & 98 \pm 1   & 52 \pm 1 & 202 \pm 1 & 78 \pm 1 & 453 \pm 1 \\                               
      27 \pm 1 & 102 \pm 1  & 53 \pm 1 & 207 \pm 1 & 79 \pm 1 & 471 \pm 1 \\                   
      28 \pm 1 & 105 \pm 1  & 54 \pm 1 & 212 \pm 1 & 80 \pm 1 & 487 \pm 1 \\                   
      29 \pm 1 & 108 \pm 1  & 55 \pm 1 & 218 \pm 1 & 81 \pm 1 & 507 \pm 1 \\                  
      30 \pm 1 & 112 \pm 1  & 56 \pm 1 & 225 \pm 1 & 82 \pm 1 & 528 \pm 1 \\                   
      31 \pm 1 & 116 \pm 1  & 57 \pm 1 & 231 \pm 1 & 83 \pm 1 & 547 \pm 1 \\                   
      32 \pm 1 & 119 \pm 1  & 58 \pm 1 & 237 \pm 1 & 84 \pm 1 & 578 \pm 1 \\                   
      33 \pm 1 & 122 \pm 1  & 59 \pm 1 & 245 \pm 1 & 85 \pm 1 & 592 \pm 1 \\                   
      34 \pm 1 & 126 \pm 1  & 60 \pm 1 & 252 \pm 1 & 86 \pm 1 & 612 \pm 1 \\                   
      35 \pm 1 & 131 \pm 1  & 61 \pm 1 & 258 \pm 1 & 87 \pm 1 & 635 \pm 1 \\                   
      36 \pm 1 & 134 \pm 1  & 62 \pm 1 & 265 \pm 1 & 88 \pm 1 & 656 \pm 1 \\                   
      37 \pm 1 & 138 \pm 1  & 63 \pm 1 & 273 \pm 1 & 89 \pm 1 & 678 \pm 1 \\                   
      38 \pm 1 & 141 \pm 1  & 64 \pm 1 & 281 \pm 1 & 90 \pm 1 & 700 \pm 1 \\                   
      39 \pm 1 & 145 \pm 1  & 65 \pm 1 & 290 \pm 1 & 91 \pm 1 & 724 \pm 1 \\                   
      40 \pm 1 & 149 \pm 1  & 66 \pm 1 & 299 \pm 1 & 92 \pm 1 & 753 \pm 1 \\                  
      41 \pm 1 & 153 \pm 1  & 67 \pm 1 & 308 \pm 1 & 93 \pm 1 & 775 \pm 1 \\                   
      42 \pm 1 & 157 \pm 1  & 68 \pm 1 & 317 \pm 1 & 94 \pm 1 & 809 \pm 1 \\                   
      43 \pm 1 & 162 \pm 1  & 69 \pm 1 & 327 \pm 1 & 95 \pm 1 & 835 \pm 1 \\                   
      44 \pm 1 & 166 \pm 1  & 70 \pm 1 & 337 \pm 1 & 96 \pm 1 & 867 \pm 1 \\                   
      45 \pm 1 & 170 \pm 1  & 71 \pm 1 & 348 \pm 1 & 97 \pm 1 & 899 \pm 1 \\                    
      46 \pm 1 & 174 \pm 1  & 72 \pm 1 & 360 \pm 1 & 98 \pm 1 & 938 \pm 1 \\                   
      47 \pm 1 & 178 \pm 1  & 73 \pm 1 & 373 \pm 1 & 99 \pm 1 & 989 \pm 1 \\                   
      48 \pm 1 & 183 \pm 1  & 74 \pm 1 & 387 \pm 1 & 100 \pm 1 & 999 \pm 1 \\                   
      49 \pm 1 & 187 \pm 1  & 75 \pm 1 & 401 \pm 1 & & \\ 
      50 \pm 1 & 192 \pm 1  & 76 \pm 1 & 417 \pm 1 & & \\ 
      \bottomrule
    \end{longtblr}
     Diese Werte werden in Abbildung (\ref{fig:Druck_unter_1_bar}) aufgetragen. 
     %Der Druck als $\ln{\frac{p}{p_0}}$ auf der y Achse und die Temperatur in $\frac{1}{T} \,\left[\frac{1}{\symup{K}}\right]$ auf der x-Achse. 


    \begin{figure}
      \centering
      \includegraphics{plot1.pdf}
      \caption{Graphische Darstellung der Messwerte aus Tabelle (\ref{tab:Druck_unter_1_bar}) mit Ausgleichsgerade.}
      \label{fig:Druck_unter_1_bar}
    \end{figure}
    Die Ausgleichsgerade hat die Form 

    \begin{equation}
      \label{eqn:Ausgleichsgerade}
      \ln{\left(\frac{p}{p_0}\right)} = m \cdot \frac{1}{T} \,\left[\frac{1}{\symup{K}}\right] + n \, . 
    \end{equation}

    Für $m$ und $n$ ergeben sich die Werte $m = (-3417 \pm 53) \, \unit{\kelvin}$
    und $n = (9 \pm 0,2)$.
    Da $p$ unter 1 bar ist, dürfen die Vereinfachungen angenommen werden, die zu Gleichung (\ref{eqn:Druck_p}) führen. Diese Formel umgestellt ist 
    \begin{equation}
      \label{eqn:Druck_p_umgestellt}
      \ln{\left(\frac{p}{p_0}\right)} = - \frac{L}{R} \cdot \frac{1}{T} \, . 
    \end{equation}
    Mithilfe der Ausgleichsgerade (\ref{eqn:Ausgleichsgerade}) wird Formel 
    (\ref{eqn:Druck_p_umgestellt}) nach 
    \begin{align}
      m &= - \frac{L}{R} \\
      \Leftrightarrow L &= - \, m \cdot R
    \end{align}
    umgestellt. Für $R = 8,3145 \, \unit[per-mode=fraction]{\joule\per\mol\per\kelvin}$ ergibt dies einen Wert von $L = (28,4 \pm 0,4) \cdot 10^3 \, \unit[per-mode=fraction]{\joule\per\mol}$.
    \\
    Zur Berechnung der inneren Verdampfungswärme $L_{\text{i}}$ wird Formel (\ref{eqn:Verdampfungswaerme}) verwendet. Diese umgestellt ergibt 
    \begin{equation}
      L_{\text{i}} = L - L_{\text{a}} \, . 
    \end{equation}
    Die äußere Verdampfungswärme $L_{\text{a}}$ wird mithilfe der ideale Gasgleichung (\ref{eqn:idealeGasgleichung}) für eine Temperatur $T_{\text{a}}$ von $373 \, \unit{\kelvin}$ wie folgt abgeschätzt
    \begin{equation}
      L_{\text{a}} = R \cdot  T_{\text{a}}\, . 
    \end{equation}
    Dies ergibt $L_{\text{a}} = 3,1013 \cdot 10³ \, \unit[per-mode=fraction]{\joule\per\mol}$. 
    Mithilfe von $L_{\text{a}}$ wird $L_{\text{i}} = (25,3 \pm 0,4) \cdot 10³ \, \unit[per-mode=fraction]{\joule\per\mol}$ bestimmt. 
    Um die Einheit der Energie $L_{\text{i}}$ in eine Energie mit Einheit Elektronenvolt pro Molekül $L_{\text{i,M}}$ umzurechnen, wird 
    \begin{equation}
      L_{\text{i,M}} = \frac{L_{\text{i}}}{\symup{N} \cdot \symup{e}}
      \label{eqn:L_Molekül}
    \end{equation}
    angewandt. $\symup{N}$ ist dabei die Avogadrokonstante $6,0221 \cdot 10^{23} \, \frac{1}{\symup{mol}}$ und $\symup{e}$ die Elementarladung mit 
    $\symup{e} = 1,6022 \cdot 10^{-19} \, \unit{\ampere\second}$. 
    Durch Gleichung (\ref{eqn:L_Molekül}) wird $L_{\text{i,M}}$ berechnet zu $L_{\text{i,M}} = (0,262 \pm 0,005) \, \symup{e}\symup{V}$.
    
    \subsection{Verdampfungswärme von Wasser von 1 bar bis 15 bar}
    
    
    
    %Siehe \autoref{fig:plot}!
%
%Alte Tabelle: 
    %\begin{table}[H]
    %  \centering
    %  \caption{Gemessener Druck $p$ bei verschiedenen Temperaturen $T$}
    %  \label{tab:Druck_unter_1_bar1}
    %  \begin{tblr}{colspec={c c c c c c}}
    %      \toprule
    %      $T \, \left[\unit{\celsius}\right]$ & $p \, \left[\unit{\milli\bar}\right]$ & $T \, \left[\unit{\celsius}\right]$ & $p \, \left[\unit{\milli\bar}\right]$ & $T \, \left[\unit{\celsius}\right]$ & $p \, \left[\unit{\milli\bar}\right]$\\ 
    %      \midrule
    %      25 \pm 1 & 95 \pm 1   & 51 \pm 1 & 197 \pm 1 & 77 \pm 1 & 438 \pm 1 \\                     
    %      26 \pm 1 & 98 \pm 1   & 52 \pm 1 & 202 \pm 1 & 78 \pm 1 & 453 \pm 1 \\                               
    %      27 \pm 1 & 102 \pm 1  & 53 \pm 1 & 207 \pm 1 & 79 \pm 1 & 471 \pm 1 \\                   
    %      28 \pm 1 & 105 \pm 1  & 54 \pm 1 & 212 \pm 1 & 80 \pm 1 & 487 \pm 1 \\                   
    %      29 \pm 1 & 108 \pm 1  & 55 \pm 1 & 218 \pm 1 & 81 \pm 1 & 507 \pm 1 \\                  
    %      30 \pm 1 & 112 \pm 1  & 56 \pm 1 & 225 \pm 1 & 82 \pm 1 & 528 \pm 1 \\                   
    %      31 \pm 1 & 116 \pm 1  & 57 \pm 1 & 231 \pm 1 & 83 \pm 1 & 547 \pm 1 \\                   
    %      32 \pm 1 & 119 \pm 1  & 58 \pm 1 & 237 \pm 1 & 84 \pm 1 & 578 \pm 1 \\                   
    %      33 \pm 1 & 122 \pm 1  & 59 \pm 1 & 245 \pm 1 & 85 \pm 1 & 592 \pm 1 \\                   
    %      34 \pm 1 & 126 \pm 1  & 60 \pm 1 & 252 \pm 1 & 86 \pm 1 & 612 \pm 1 \\                   
    %      35 \pm 1 & 131 \pm 1  & 61 \pm 1 & 258 \pm 1 & 87 \pm 1 & 635 \pm 1 \\                   
    %      36 \pm 1 & 134 \pm 1  & 62 \pm 1 & 265 \pm 1 & 88 \pm 1 & 656 \pm 1 \\                   
    %      37 \pm 1 & 138 \pm 1  & 63 \pm 1 & 273 \pm 1 & 89 \pm 1 & 678 \pm 1 \\                   
    %      38 \pm 1 & 141 \pm 1  & 64 \pm 1 & 281 \pm 1 & 90 \pm 1 & 700 \pm 1 \\                   
    %      39 \pm 1 & 145 \pm 1  & 65 \pm 1 & 290 \pm 1 & 91 \pm 1 & 724 \pm 1 \\                   
    %      40 \pm 1 & 149 \pm 1  & 66 \pm 1 & 299 \pm 1 & 92 \pm 1 & 753 \pm 1 \\                  
    %      41 \pm 1 & 153 \pm 1  & 67 \pm 1 & 308 \pm 1 & 93 \pm 1 & 775 \pm 1 \\                   
    %      42 \pm 1 & 157 \pm 1  & 68 \pm 1 & 317 \pm 1 & 94 \pm 1 & 809 \pm 1 \\                   
    %      43 \pm 1 & 162 \pm 1  & 69 \pm 1 & 327 \pm 1 & 95 \pm 1 & 835 \pm 1 \\                   
    %      44 \pm 1 & 166 \pm 1  & 70 \pm 1 & 337 \pm 1 & 96 \pm 1 & 867 \pm 1 \\                   
    %      45 \pm 1 & 170 \pm 1  & 71 \pm 1 & 348 \pm 1 & 97 \pm 1 & 899 \pm 1 \\                    
    %      46 \pm 1 & 174 \pm 1  & 72 \pm 1 & 360 \pm 1 & 98 \pm 1 & 938 \pm 1 \\                   
    %      47 \pm 1 & 178 \pm 1  & 73 \pm 1 & 373 \pm 1 & 99 \pm 1 & 989 \pm 1 \\                   
    %      48 \pm 1 & 183 \pm 1  & 74 \pm 1 & 387 \pm 1 & 100 \pm 1 & 999 \pm 1 \\                   
    %      49 \pm 1 & 187 \pm 1  & 75 \pm 1 & 401 \pm 1 & & \\ 
    %      50 \pm 1 & 192 \pm 1  & 76 \pm 1 & 417 \pm 1 & & \\ 
    %               
    %      \bottomrule
    %  \end{tblr}
    %\end{table}