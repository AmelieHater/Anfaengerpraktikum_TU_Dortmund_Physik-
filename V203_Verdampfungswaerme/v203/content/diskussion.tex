\section{Diskussion}
\label{sec:Diskussion}
Die relative Abweichunge zwischen dem theoretischen und dem experimentellen Wert wird bestimmt durch
$$\text{rel. Abweichung} = \frac{|\text{exp. Wert} - \text{theo. Wert}|}{\text{theo. Wert}}\,.$$

- Evakuierung von 47 mbar nur erreicht am Anfang, weißt darauf hin, dass gesamte Apperatur im 1. Teilexperiment undicht ist \\
$\Rightarrow$ wird Messdaten sehr verfäscht haben \\

- Ablesen des Themomenters nach langem Draufstarren schwierig, Ableseparalaxe nicht auszuschließen \\

- durch Konzept von eine Person ließt ab und eine Person schreibt auf, ist Zeitverzögerung im Spiel, gerade beim Ablesen des korrekten Drucks bei hohen Temperaturen 
(sehr schnelle Steigung da p ja exponentiell wächst)\\

- Literaturwert für die Verdampfungswärme von Wasser ist $L_{\text{lit}} = 40,8 \, \unit[per-mode=fraction]{\kilo\joule\per\mol}$ \cite{L_Literatur}.\\

- unserer ist $L_{\text{exp}} = (28,4 \pm 0,4) \, \unit[per-mode=fraction]{\kilo\joule\per\mol}$ \\

- relative Abweichung daher: 0,303912 bzw. 30,39 \%




