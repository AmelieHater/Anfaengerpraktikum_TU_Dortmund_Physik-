\section{Diskussion}
\label{sec:Diskussion}
Die relative Abweichung zwischen dem theoretischen und dem experimentellen Wert wird bestimmt durch
$$\text{rel. Abweichung} = \frac{|\text{exp. Wert} - \text{theo. Wert}|}{\text{theo. Wert}}\,.$$
Der experimentell bestimmte Wert $L_{\text{exp}}$ ist $(28,4 \pm 0,4) \, \unit[per-mode=reciprocal]{\kilo\joule\per\mol}$. Der theoretische Wert ist 
$L_{\text{lit}} = 40,8 \, \unit[per-mode=reciprocal]{\kilo\joule\per\mol}$ \cite{L_Literatur}.
Dies führt zu einer relativen Abweichung von 30,39 \%. 
Ein Grund für diese etwas höhere Abweichung könnte die Undichte der Apparatur sein. Ursprünglich sollte ein Enddruck von ungefähr $30 \,\unit{\milli\bar}$ erreicht werden. Allerdings wird in diesem Versuch ein Enddruck von $47 \,\unit{\milli\bar}$,
was ungewöhnlich hoch ist. Dies kann darauf hinweisen, dass die Apparatur undicht sein könnte. 
Dieser Umstand könnte die Messdaten in signifikantem Maß verfälscht haben. Außerdem steigt die Temperatur nach Einschalten der Heizhaube sehr schnell an, sodass
durch die Reaktionszeiten der beiden messenden Personen eine Verfälschung entstanden sein könnte.
Insbesondere könnte dies im höheren Bereich der Temperatur aufgetreten sein, da der Druck in dem Bereich am schnellsten ansteigt.
Zudem ist das Thermometer nur aus bestimmten Winkeln gut ablesbar, 
was dafür gesorgt haben könnte, dass einige Messdaten mit großem Fehler aufgenommen wurden. \\
Im zweiten Teil des Experiments wurde die Verdampfungswärme in einem Bereich von 1 bis 15 bar bestimmt. Durch die Wurzel kommen mathematisch betrachtet 2 Lösungsfunktionen
in betracht, physikalisch ist allerdings nur die Lösung mit positivem Wurzelvorzeichen denkbar. Dies ist der Fall, da die Verdampfungswärme mit steigender Temperatur 
weniger werden sollte, da sie im kritischen Punkt 0 sein sollte, da dort flüssig und gasförmig nicht mehr voneinander getrennt werden können.





