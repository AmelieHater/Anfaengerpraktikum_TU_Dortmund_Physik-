\section{Diskussion}
\label{sec:Diskussion}
Die relative Abweichunge zwischen dem theoretischen und dem experimentellen Wert wird bestimmt durch
$$\text{rel. Abweichung} = \frac{|\text{exp. Wert} - \text{theo. Wert}|}{\text{theo. Wert}}\,.$$
Der experimentell bestimmte Wert $L_{\text{exp}}$ ist $(28,4 \pm 0,4) \, \unit[per-mode=reciprocal]{\kilo\joule\per\mol}$. Der theoretische Wert ist 
$L_{\text{lit}} = 40,8 \, \unit[per-mode=reciprocal]{\kilo\joule\per\mol}$ \cite{L_Literatur}.
Dies führt zu einer relativen Abweichung von 30,39 \%. 
Ein Grund für diese etwas höhere Abweichung könnte die Undichte der Apperatur sein. Ein Enddruck von $47 \,\unit{\milli\bar}$, wenn ursprünglich ein 
Enddruck von ungefähr $30 \,\unit{\milli\bar}$ erreicht werden sollte, ist ungewöhnlich hoch und kann darauf hinweisen, dass die Apperatur undicht sein könnte.
Dieser Umstand könnte die Messdaten in signifikantem Maß verfälscht haben. Außerdem steigt die Temperatur nach Einschalten der Heizhaube sehr schnell an, sodass
durch die Reaktionszeiten der beiden messenden Personen eine Verfälschung entstanden sein könnte, vor allen Dingen im Bereich der höheren Temperaturen, in 
denen der Druck am schnellsten ansteigt. Zudem ist das Thermometer nur aus bestimmten Winkeln gut ablesbar, 
was dafür gesorgt haben könnte, dass einige Messdaten mit großem Fehler aufgenommen wurden. \\
Im zweiten Teil des Experiments wurde die Verdampfungswärme in einem Bereich von 1 bis 15 bar bestimmt. Durch die Wurzel kommen mathematisch betrachtet 2 Lösungsfunktionen
in betracht, physikalisch ist allerdings nur die Lösung mit positivem Wurzelvorzeichen denkbar. Dies ist der Fall, da die Verdampfungswärme mit steigender Temperatur 
weniger werden sollte, da sie im kritischen Punkt 0 sein sollte, da dort flüssig und gasförmig nicht mehr voneinander getrennt werden können.





