\section{Diskussion}
\label{sec:Diskussion}
Die Abweichung der Steigung der beiden Ausgleichgerade für die Bestimmung der Strömungsgeschwindigkeit als Funktion des Dopplerwinkels ist mit $0 \, \%$
auffällig gering. Die Bestimmung des Strömungsprofils hingegen war fehlerbehafteter. Theoretisch sollte eine laminare Strömung im Rohr vorhanden sein, 
was bei den Abbildungen für einen paraboloiden Verlauf der Messwerte sorgen sollte. Dieser ist allerdings allein in Rohr 1 für eine Pumpgeschwindigkeit von 
$3 \, \unit[per-mode=fraction]{\liter\per\meter}$ zu sehen. Für eine Pumpgeschwindigkeit von $6 \, \unit[per-mode=fraction]{\liter\per\meter}$
ist die Kurve deutlich flacher und weniger paraboloidförmig. Auch bei Rohr 2 ist ein ähnlicher 
Unterschied zwischen de verschiedenen Pumpgeschwindigkeiten zu sehen. Bei der Pumpgeschwindigkeit von 
$6 \, \unit[per-mode=fraction]{\liter\per\meter}$ stagniert die Momentangeschwindigkeit über eine 
große Tiefe hinweg. Hier liegt die Vermutug nahe, dass das Auflösungsvermögen der Messapperatur zu gering war,
um die feineren Unterschiede aufzulösen. 
Anfänglich waren Messschwierigkeiten aufgrund eines Wackelkontakts der Ultraschallsonde vorhanden, wodurch die anzulesende Messgröße sehr schnell sprang. 
Sobald diese behoben wurden, sprang die abzulesende Messgröße nicht mehr und die experimentellen Fehler wurden durch den sich nicht veränderten Versuchsaufbau
verhindert. Zusätlich wurden die Messdaten durch einen Computer aufgenommen, wodurch keine Ablesefehler entsehen
konnten. Dies könnte die auffällig geringe Abweichung der Bestimmung der Strömunggeschwindigkeit erklären. 