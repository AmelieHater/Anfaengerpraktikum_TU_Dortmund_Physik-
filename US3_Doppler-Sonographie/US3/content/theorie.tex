\section{Zielsetzung}
\label{sec:Zielsetzung}
\nocite{anleitungUS3}
Das Ziel dieses Versuchs ist die Untersuchung der charakteristischen Eigenschaften von Strömungen anhand des Impuls-Echo-Verfahren.

\section{Theorie}
\label{sec:Theorie}
Die menschliche Hörschwelle liegt zwischen ca. $16\,\unit{\hertz}$ und ca. $20\,\unit{\kilo\hertz}$. Der Ultraschall Frequenzbereich
liegt oberhalb der Hörschwelle bei ca. $20\,\unit{\kilo\hertz}$ bis ca. $1\,\unit{\giga\hertz}$. Von einem Hyperschall wird gesprochen, wenn
der Frequenzbereich über $1\,\unit{\giga\hertz}$ liegt. Der Infraschall Frequenzbereich liegt unterhalb der Hörschwelle. 
Der Doppler-Effekt beschreibt die Änderung der Frequenz bei relativer Bewegung zwischen einem Beobachter und einer Schallquelle. 
Wenn sich die Quelle auf den Beobachter zu bewegt, wird die Frequenz $\nu_0$ zu einer höheren Frequenz $\nu_{\text{gr}}$ und wenn die
Qulle sich vom Beobachter entfernt, sinkt die Frequenz $\nu_0$ auf einer niedrigeren Frequenz $\nu_{\text{kl}}$. Diese Beziehungen lassen sich mit
der Gleichung
\begin{equation}
    \nu_{\text{gr/kl}} = \frac{\nu_0}{1 \mp \frac{v}{c}}
    \label{eqn:Frequenz_Quelle-Beobachter}
\end{equation}
beschreiben, wobei $v$ die Geschwindigkeit des Objekts und $c$ die Schallgeschwindigkeit ist. Falls die Quelle stationär bleibt und der Beobachter sich der Quelle nähert, dann erhöht sich die Frequenz $\nu_0$ auf eine höhere Frequenz
$\nu_{\text{h}}$. Entfernt sich der Beobachter von der Quelle weg, dann sinkt die Frequenz $\nu_0$ auf eine Frequenz $\nu_{\text{n}}$. Die Veränderungen lassen sich
durch folgende Beziehung beschreiben 
\begin{equation}
    \nu_{\text{h/n}} = \nu_0 \left( 1 \pm \frac{v}{c}\right)\,.
    \label{eqn:Frequenz_Beobachter-Quelle}
\end{equation}
Die Frequenzverschiebung $\Delta \nu$ wird mithilfe der Winkel $\alpha$ und $\beta$ ermittelt, welche die Winkel zwischen der Geschwindigkeit $v$ und der
Wellennormalen der einlaufenden bzw. auslaufenden Welle beschreibt. Diese Beziehung lautet 
\begin{equation}
    \Delta \nu = \nu_0 \frac{v}{c} \left(\cos \alpha + \cos \beta\right)\,.
    \label{eqn:Frequenzverschiebung}
\end{equation}
Bei diesem Versuch wird das Impuls-Echo-Verfahren verwendet, bei dem die Winkel $/alpha$ und $\beta$ identisch sind. Daraus folgt für die
Frequenzverschiebung
\begin{equation}
    \Delta \nu = 2 \nu_0 \frac{v}{c}\cos \alpha \,.
    \label{eqn:Frequenzverschiebung_identisch}
\end{equation}
Die Erzeugung von Ultraschall ist unter anderem durch die Methode des reziproken piezo-elektrischen Effekts möglich. Hierfür wird ein piezoelektrischer Kristall in ein 
elektrisches Wechselfeld eingeführt. Der Kristall wird zum schwingen angeregt, wenn eine polare Achse des Kristalls zum elektrischen Feld zeigen und strahlt Ultraschallwellen währenddessen ab.

\subsection{Vorbereitungsaufgaben}
\label{sec:Vorbereitungsaufgaben}
Zur Vorbereitung soll der Dopplerwinkel von drei verschiedenen Prismenwinkeln $\theta$ berechnet werden. Diese lassen sich
mithilfe der Gleichung \ref{eqn:Dopplerwinkel} bestimmen und sind in der Tabelle \ref{tab:Vorbereitung} aufgelistet. 
\begin{table}[H]
    \centering
    \caption{Berechnete Dopplerwinkel zu drei verschiedene Prismenwinkel mit den Schallgeschwindigkeiten $c_{\text{L}}=1800\,\unit[per-mode=fraction]{\metre\per\second}$ und $c_{\text{P}}=2700\,\unit[per-mode=fraction]{\metre\per\second}$.}
    \label{tab:Vorbereitung}
    \begin{tblr}{colspec={c c}}
        \toprule
        $\theta\,[°]$ & $\alpha\,[°]$ \\
        \midrule
        15 & 80,06 \\
        30 & 70,53 \\
        60 & 54,74 \\
        \bottomrule
    \end{tblr}
  \end{table}
Außerdem sollen die Tiefeneinstellungen bestimmt werden, bei denen die Flussgeschwindigkeit der drei verschiedenen Röhren gemessen werden kann. Diese werden durch einen
Dreisatz bestimmt, da bei Acryl für die Tiefeneinstellung $4\,\unit{\micro\second}\,\,\widehat{=}\,\, 10\,\unit{\milli\metre}$ und für die Dopplerflüssigkeit $4\,\unit{\micro\second} \,\,\widehat{=}\,\, 7\,\unit{\milli\metre}$ 
gilt. 
\begin{table}[H]
    \centering
    \caption{Berechnete Tiefeneinstellungen für drei verschiedene Röhreninnendruchmessern für Acryl und der Dopplerflüssigkeit.}
    \label{tab:Vorbereitung}
    \begin{tblr}{colspec={c |c c c}}
        \toprule
        & $[\unit{\micro\second}]$  & $[\unit{\micro\second}]$ & $[\unit{\micro\second}]$ \\
        & bei $7\,\unit{\milli\metre}$ & bei $10\,\unit{\milli\metre}$ & bei $16\,\unit{\milli\metre}$\\
        \midrule
        Acryl & 2,8 & 4 & 6,4 \\
        Dopplerflüssigkeit & 4 & 5,714 & 9,143\\
        \bottomrule
    \end{tblr}
  \end{table}