\section{Diskussion}
\label{sec:Diskussion}
Die relative Abweichung zwischen dem theoretischen und dem experimentellen Wert wird bestimmt durch
$$\text{rel. Abweichung} = \frac{|\text{exp. Wert} - \text{theo. Wert}|}{\text{theo. Wert}}\,.$$
In der Tabelle (\ref{tab:relativeAbweichung_1}) sind die experimentellen und theoretischen Werte sowie deren relative Abweichungen der 
Wheatstonschen Brücke, Kapazitätsmessbrücke, Induktivitätsmessbrücke und der Maxwell-Brücke aufgelistet.
\begin{table}[H]
    \centering
    \caption{Relative Abweichung der verschiedenen Brückenschaltungen.}
    \label{tab:relativeAbweichung_1}
    \begin{tblr}{colspec={c | c || c}}
        \toprule
        exp. & theo. & rel. Abweichung\\
        \midrule
        \SetCell[c=3]{c} Wheatstonesche Brücke & &\\
        \midrule
        $R_{13,\text{exp.}} = \left(305,2\pm0,9 \right)\,\unit{\ohm}$ & $R_{13,\text{theo.}} = 319,5\,\unit{\ohm}$ & $4,47\,\%$\\ 
        $R_{14,\text{exp.}} = \left(904,1\pm2,6 \right)\,\unit{\ohm}$ & $R_{14,\text{theo.}} = 900\,\unit{\ohm}$& $0,46\,\%$ \\
        \midrule
        \SetCell[c=3]{c} Kapazitätsmessbrücke & &\\
        \midrule
        $C_{15,\text{exp.}}= \left( 771,9\pm2,5 \right)\unit{\nano\farad}$ & $C_{15,\text{theo.}}= 652\,\unit{\nano\farad}$& $18,4\,\%$\\
        $ R_{15,\text{exp.}} = \left(460,6\pm1,3\right)\,\unit{\ohm}$ &$ R_{15,\text{theo.}} = 473\,\unit{\ohm}$ &$2,62\,\%$ \\
        $C_{8,\text{exp.}}= \left( 533,0\pm1,7 \right)\,\unit{\nano\farad}$ & $C_{8,\text{theo.}}= 294,1\,\unit{\nano\farad}$ & $81,2\,\%$\\
        $R_{8,\text{exp.}} = \left( 670,8\pm1,9 \right)\,\unit{\ohm}$ & $ R_{8,\text{theo.}} = 564\,\unit{\ohm}$ & $18,94\,\%$ \\
        \midrule
        \SetCell[c=3]{c} Induktivitätsmessbrücke & &\\
        \midrule
        $L_{19,\text{exp.}} = \left( 77,68\pm0,26 \right)\,\unit{\milli\henry}$ & $L_{19,\text{theo.}} = 26,96\,\unit{\milli\henry}$ & $188,1\,\%$\\
        $R_{19,\text{exp.}} = \left( 240,50\pm0,80  \right)\,\unit{\ohm}$ & $ R_{19,\text{theo.}} = 108,7\,\unit{\ohm}$ & $121,3\,\%$ \\
        $L_{16,\text{exp.}} = \left( 159,30\pm0,50 \right)\,\unit{\milli\henry}$ & $L_{16,\text{theo.}} = 132,71\,\unit{\milli\henry}$ & $20,02\,\%$ \\
        $R_{16,\text{exp.}} = \left( 92,77\pm0,28 \right)\,\unit{\ohm}$ & $ R_{16,\text{theo.}} = 411,2\,\unit{\ohm}$ & $77,44\,\%$\\
        \midrule
        \SetCell[c=3]{c} Maxwell-Brücke & &\\
        \midrule
        $L_{19,\text{exp.}} = 28,97\,\unit{\milli\henry}$ & $L_{19,\text{theo.}} = 26,96\,\unit{\milli\henry}$ & $7,46\,\%$\\
        $R_{19,\text{exp.}} = \left( 111,51\pm0,32   \right)\,\unit{\ohm}$ & $ R_{19,\text{theo.}} = 108,7\,\unit{\ohm}$ & $2,59\,\%$ \\
        $L_{16,\text{exp.}} =  134,27\,\unit{\milli\henry}$ & $L_{16,\text{theo.}} = 132,71\,\unit{\milli\henry}$ & $1,18\,\%$ \\
        $R_{16,\text{exp.}} = \left( 414,4\pm1,2 \right)\,\unit{\ohm}$ & $ R_{16,\text{theo.}} = 411,2\,\unit{\ohm}$ & $0,78\,\%$\\
        \bottomrule
    \end{tblr}
  \end{table}
Diese Abweichungen könnten an der ungenauen Bestimmung von $R_3$ und $R_4$ liegen. 
Hierfür muss die Brückenspannung minimiert werden, welche mithilfe vom Ablesen am Oszilloskop minimiert wurde. Allerdings entsteht bei der
Minimierung ein Rauschen, was das Ablesen ungenau machen könnte. Zudem ist häufiger das Problem aufgetreten, dass beim Berühren der Kabel oder
des Potentiometers sich die Abbildung auf dem Oszilloskop geändert hat, was ebenfalls die teilweise großen Abweichungen erklären könnte. 
\\
Bei der Wien-Robinson-Brücke fällt auf, dass die Messwerte mehr von der Theoriekurve abweichen, desto weiter sich die Frequenz von $f_0$ entfernt. 
Allerdings lässt sich mit dem geringen Klirrfaktor von $k=0,10$ die hohe Genauigkeit im Frequenzbereich um $f_0$ bestätigen.



% Bei der Wheatstoneschen Brücke ist zwischen dem experimentellen Wert $R_{13,\text{exp.}} = \left(305,2\pm0,9 \right)\,\unit{\ohm}$
% und dem theoretischen Wert $R_{13,\text{theo.}} = 319,5\,\unit{\ohm}$ eine Abweichung von $4,47\,\%$. Für den zweiten unbekannten Widerstand
% beträgt der experimentelle Wert $ R_{14,\text{exp.}} = \left(904,1\pm2,6 \right)\,\unit{\ohm}$ und ther Theoriewert $R_{14,\text{theo.}} = 900\,\unit{\ohm}$. 
% Somit ist hier eine Abweichung von $0,46\,\%$ vorhanden. Diese Abweichungen könnten an der ungenauen Bestimmung von $R_3$ und $R_4$ liegen. 
% Hierfür musste die Brückenspannung minimiert werde, welche mithilfe vom Ablesen am Oszilloskop minimiert wurde. Allerdings entsand bei der
% Minimierung ein Rauschen, was das Ablesen ungenau machen könnte. Diese Erklärung gilt auch für die folgenden Abweichung bei der Kapazität-, Induktivitätsmessbrücke 
% und Maxwell-Brücke. Für die Kapazitätsmessbrücke ergibt sich die Kapazität $C_{15,\text{exp.}}= \left( 771,9\pm2,5 \right)\unit{\nano\farad}$ und die theoretische
% Kapazität beträgt $C_{15,\text{theo.}}= 652\,\unit{\nano\farad}$, wodurch eine Abweichung von $18,4\,\%$ vorhanden ist. Der berechnete Widerstand ist
% $ R_{15,\text{exp.}} = \left(460,6\pm1,3\right)\,\unit{\ohm}$, welche um $2,62\,\%$ vom Theoriwert $ R_{15,\text{theo.}} = 473\,\unit{\ohm}$ abweicht.
% Bei der zweiten Durchführung weicht der Experimentalwert $C_{8,\text{exp.}}= \left( 533,0\pm1,7 \right)\,\unit{\nano\farad}$ von der theoretischen Kapazität
% $C_{8,\text{theo.}}= 294,1\,\unit{\nano\farad}$ um $81,2\,\%$ ab. Der Widerstand hat eine Abweichung von $18,94\,\%$ bei einem Experimentalwert von $R_{8,\text{exp.}} = \left( 670,8\pm1,9 \right)\,\unit{\ohm}$
% und einem Theoriwert von $ R_{8,\text{theo.}} = 564\,\unit{\ohm}$. 