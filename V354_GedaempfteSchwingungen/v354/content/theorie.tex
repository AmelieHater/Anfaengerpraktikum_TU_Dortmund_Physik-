\section{Zielsetzung}
\label{sec:Zielsetzung}
Das Ziel dieses Versuches ist sich mit verschiedene gedämpften und erzwungen Schwingungen innerhalb einer Schaltung
 bestehend aus Widerständen, Kondensatoren und Spulen, auseinanderzusetzen. Insbesondere wird sich mit einer gedäpften Schwingung, dem aperiodischen 
 Grenzfall und Frequenzabhängigkeit der Kondensatorspannung beschäftigt. 
\section{Theorie}
\label{sec:Theorie}
Ein ungedämpfter Schwingkreis besteht aus einer Spule mit Induktivität $L$ und einem Kondensator mit Kapazität $C$. In diesem gilt Energieerhaltung, was bedeutet, 
dass die Gesamtenergie, die im Schwingkreis gespeichert ist, konstant bleibt. Diese Gesamtenergie setzt sich aus der Energie zusammen, die im Magnetfeld der Spule
gespeichert ist und der Energie, die im elektrischen Feld des Kondensators gespeichert ist. Diese beiden Energieformen oszillieren verlustfrei im Schwingkreis. 

\subsection{Gedämpfte Schwingungen}
In einem gedämpften Schwingkreis ist zusätzlich zu der Spule und dem Kondensatir ein ohmscher Widerstand $R$ eingebaut, an dem elektrische Energie konstant in 
Wärmeenergie umgewandelt wird und dadurch das System verlässt. Daher ist keine Energieerhaltung im gedämpften Schwingkreis gegeben. Allerdings findet trotzdessen eine 
Oszillation der einzelnen Energien statt, die Gesamtenergie nähert sich allerdings stetig der $0$ an. 
Das Verhalten des gedämpften Schwingkreises lässt sich durch die Differentialgleichung 
\begin{equation}
    \ddot{I} + \frac{R}{L} \cdot \dot{I} + \frac{1}{L C} \cdot I = 0 
\end{equation}
beschreiben. $I$ ist dabei der im Schwingkreis fließende Strom. 
Diese Gleichung wird durch die Funktion  
\begin{equation}
    I = \symup{e}^{-2\pi\mu t} \cdot \left(B_1 \cdot \symup{e}^{2\pi\nu \symup{i}t}  + B_2 \cdot \symup{e}^{- 2\pi\nu \symup{i}t} \right)
\end{equation}
gelöst. Für diese Funktion wurden die Abkürzungen 
\begin{equation*}
2 \pi \mu \coloneqq \frac{R}{2 L}  \,\,\text{und}\,\, 2 \pi \nu \coloneqq \sqrt{\frac{1}{LC} - \left(\frac{R}{2L}\right)²}
\end{equation*}

