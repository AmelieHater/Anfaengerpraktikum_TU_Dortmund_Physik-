\section{Durchführung}
\label{sec:Durchführung}
Zur Aufnahme von Messwerten stehen verschiedene ohmsche Widerstände ($R_1$, $R_2$, $R_3$),
 eine Spule und ein Kondensator zur Verfügung. Außerdem werden die Spannungsverläufe 
 auf einem Oszilloskop dargestellt und ein Generator zur Erzeugung verschiedener 
 Spannungen ist vorhanden. 
\subsection{Amplitude einer gedämpften Schwingung}
Im ersten Versuchsteil wird eine Schwingung im gedämpften Schwingkreis gemessen. Für
diese Messung wird eine vom Generator erzeugte Rechteckspannung in den Schwingkreis 
mit $R_1$ als ohmschen Widerstand geleitet. Auf dem an den Schwingkreis angeschlossenen
Oszilloskop wird dann die oszillierende Schwingung scharfgestellt und die 
Amplituden in Abhängigkeit von der Zeit abgelesen. 
\subsection{Aperiodischen Grenzfall}
Im zweiten Teil des Versuches wird statt $R_1$ der verstellbare ohmsche Widerstand $R_3$
verwendet. Der Generator erzeugt auch in diesem Teil eine Rechteckspannung, die in den 
Schwingkreis geleitet wird. Auf dem Oszilloskop wird ebenso wie beim 1. Teil der zeitliche
Spannungsverlauf abgelesen. Zu Beginn der Messung ist der verstellbare Widerstand auf 
dem höchsten Wert eingestellt. Dann wird der Widerstand stetig verkleinert bis die Kurve
des aperiodischen Grenzfalls auf dem Oszilloskop zu sehen ist. Der Widerstand, bei dem 
dies der Fall ist, wird notiert. 
\subsection{Frequenzabhängigkeit der Kondensatorspannung}
Beim letzten Teil der Versuches wird die Kondensatorspannung auf eine Frequenzabhängigkeit 
untersucht. Nun wird $R_2$ verwendet und eine sinusförmige Spannung durch den Generator
erzeugt, um diese in den gedämpften Schwingkreis zu leiten. Auf dem Oszilloskop werden
zwei Spannungsverläufe aufgezeichenet, einmal den der Erregerspannung und den der 
Kondensatorspannung. Zu Beginn wird versucht die Erregerfrequenz zu finden, die für die 
höchste Kondensatorspannung sorgt. Im folgenden wird die Kondensatorspannung 
in Abhängigkeit von der Erregerfrequenz gemessen und notiert. Dabei ist darauf zu achten, 
dass die Erregerfrequenzen möglichst gleichmäßig um die Erregerfrequenz, die den 
höchsten Ausschlag der Kondensatorspannung verursacht, gewählt werden. 