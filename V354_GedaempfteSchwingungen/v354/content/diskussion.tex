% Diskussion:
% R_eff_exp: 162.3+/-0.5 in ohm und R_1: 67.20+/-0.10 in ohm 
% rel. Abw. R eff: 1.416+/-0.008
% T_ex_exp: 415.66 in s und T_ex_theo: 502.0+/-2.0
% rel. Abw. Abklingzeit: 0.17199203187250992
% R_ap_exp: 4500 in ohm und R_ap_theo: 5723+/-9 in ohm 
% rel. Abw. R ap: 0.2138+/-0.0013
% Resonanzüberhöhung q_exp: 3.140+/-0.023 und q_theo: 4.196+/-0.008
% rel. Abw. Maximum(Resonanzüberhöhung q): 0.2518+/-0.0014
% delta_freq_exp: 9310.0 in Hz und delta_freq_theo: 6434+/-20 in Hz 
% rel. Abw. Breite: 0.447+/-0.004
\section{Diskussion}
\label{sec:Diskussion}
Die relative Abweichung zwischen dem theoretischen und dem experimentellen Wert wird bestimmt durch
$$\text{rel. Abweichung} = \frac{|\text{exp. Wert} - \text{theo. Wert}|}{\text{theo. Wert}}\,.$$

Für die gedämpfte Schwingung beträgt der experimentelle, effektive Widerstand $R_{\text{eff,exp.}} = \left(162,3 \pm 0,5\right)\,\unit{\ohm}$.
Bei dieser Durchführung wird der Widerstand $R_1 = \left(67,20 \pm 0,10\right)\,\unit{\ohm}$ verwendet. Somit ergibt sich eine relative Abweichung 
von $141,6\,\%$. Diese große Abweichung könnte daran liegen, dass der Innenwiderstand des Frequenzgenerators vernachlässigt wird. Somit wäre 
der theoretische, effektive Widerstand größer und demnach die Abweichung kleiner. Für die Abklingdauer ergeben sich die Werte $T_{\text{ex,exp.}}= 415,66\,\unit{\micro\second}\,.$ und
$T_{\text{ex,theo.}}= \left(502\pm2\right)\,\unit{\micro\second}$. Hier beträgt die Abweichung $17,2\,\%$. Außerdem gab es viele Messwerte, die knapp über bzw. knapp unter einem Messstrich
liegen und dann beide auf denselben Messstrich gerundet werden. Dadurch können Fehler zustande gekommen sein. Dieser Ablesefehler ist ebenfalls für den dritten 
Versuchsteil relevant. \\
Zur Bestimmung des aperiodischen Widerstands ergibt sich für den experimentellen Wert $R_{\text{ap,exp.}} = 4500 \,\unit{\ohm}$ und der theoretische
Wert beträgt $R_{\text{ap,theo.}}\left(5723\pm9\right)\,\unit{\ohm}$. Demnach ist eine relative Abweichung von $21,4\,\%$ vorhanden. Auf der Abbildung des Oszilloskops
ist der genaue Übergang zwischen dem aperiodischen Grenzfall und einem Überschlag nicht genau ablesbar und muss abgeschätzt werden. Diese Ungenauigkeit könnte die relative Abweichung 
des aperiodischen Widerstands erklären.\\
Bei der dritten Durchführung ergeben sich für die Güten $q_{\text{exp.}} = 3,140 $ und $q_{\text{theo.}} =\left(4,196\pm0,008\right)$. Die relative Abweichung
von diesen Werten beträgt $25,2\,\%$. Außerdem ergeben sich für die Breite der Resonanzkurven $\Delta f_{\text{exp.}} =9310\,\unit{\hertz}$ und
$\Delta f_{\text{theo.}} =\left(6434\pm20\right)\,\unit{\hertz}$. Für die Breite der Resonankurve ergibt sich eine relative Abweichung von $44,7\,\%$.
Hier könnten die Abweichungen unter anderem durch den oben beschriebenen Ablesefehler entstehen. 
