% rel Abweichugn n_s reduziert: 43.07939064269387
% rel Abweichugn n_p reduziert: 34.68266058748819
% rel Abweichugn n_p: 11.78050306980616
% rel Abweichugn n_s: 43.07939064269387
\section{Diskussion}
\label{sec:Diskussion}
Die relative Abweichung zwischen dem theoretischen und dem experimentellen Wert wird bestimmt durch
$$\text{rel. Abweichung} = \frac{|\text{exp. Wert} - \text{theo. Wert}|}{\text{theo. Wert}}\,.$$
Die relative Abweichung zwischen dem berechneten Brechungsindex $\overline{n_{\text{s}}}$ und dem theoretischen Brechungsindex $n$
beträgt $47,26\,\%$. Bei dem p-polarisiertem Laser ist die relative Abweichung $34,68\,\%$. Diese großen Abweichungen lassen sich 
aufgrund des Versuchsaufbau erklären. Während der Durchführung ist bereits aufgefallen, dass die Werte nicht dem Verlauf der Theoriekurve
entsprechen, weswegen eine neue Messreihe mit einem anderem Polarisationsfilter durchgeführt wurde. Bei der verwendeten Messreihe, sahen vor Allem die Werte des s-polarisiertem 
Lasers grob nach dem Verlauf der Theoriekurve aus, wodurch sich für diese Messreihe entschieden wurde. Bei der Auswertung fällt auf, dass die unbereinigten Brechungsindizes bei dem 
p-polarisiertem Laser einen größeren Fehlerbereich nachweist. Daher könnte eine weitere Erklärung für die hohen Abweichungen, ein nicht ideal funktionierender
Polarisationsfilter sein. 
