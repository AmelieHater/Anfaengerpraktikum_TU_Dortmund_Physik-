
% Brechungsindex theorie $n = 3,73$


\nocite{anleitungV407}
\section{Auswertung}
\label{sec:Auswertung}
Die zuerst gemessenen Werte des Nullstroms $I_0$ und des Dunkelstroms $I_{\text{D}}$ lauten
\begin{align*}
  I_0 &= 460\,\unit{\micro\ampere}\\
  I_{\text{D}} &= 2,8\,\unit{\nano\ampere}\,.
\end{align*}
Während der Messung des Dunkelstroms, ist das Photoelement so positioniert worden, dass es einer maximalen 
Störung der Lichtquellen ausgesetzt ist. Bei der Durchführung treffen im Allgemeinen nur kleinere Störungen auf
das Photoelement. Zusätzlich befindet sich die maximale Störung in einem kleinen Bereich, weswegen der Dunkelstrom im weiteren Verlauf der Auswertung vernachlässigt wird.
\\In der Tabelle \ref{tab:Messwerte} sind die gemessenen Photoströme in Abhängigkeit des Einfallswinkels $\alpha$ aufgelistet.
Um die Brechungsindizes für s- und p-polarisiteres Licht zu bestimmen, werden die Gleichungen (\ref{eqn:E_r_senkrecht}) und (\ref{eqn:E_r_parallel}) nach $n_{\text{s}}$ und $n_{\text{p}}$
umgestellt. Daraus ergeben sich
\begin{align}
  n_{\text{s}} &= \sqrt{\frac{E^2 - 2E\cos(2\alpha) + 1}{E^2- 2E + 1}} \label{eqn:n_s} \qquad {\text{und}}\\
  n_{\text{p}} &= \sqrt{\left(\frac{1+E}{1-E}\right)^2 \frac{1}{2\cos^2(\alpha)} + \sqrt{\frac{(1+E)^4}{4\cos^4(\alpha)(1-E)^4}- \frac{1}{(1-E)^2} \tan^2(\alpha)}} \label{eqn:n_p}\,.
  % n_{\text{p}} &= \sqrt{\frac{(1+E)^2}{(1-E)^2}\frac{1}{2\cos^2(\alpha)} + \frac{\sqrt{(1+E)^2-4\cos^2(\alpha)(1-E)^2\sin^2(\alpha)}}{2\cos^2(\alpha)(1-E)^2}}
\end{align}
Hierfür gilt $$E = \frac{E_{\text{ref}}}{E_{\text{ein}}} = \frac{I_{\text{ref}}(\alpha)}{I_0}\,.$$
\begin{table}[H]
  % \centering
  \caption{Gemessene Photoströme bei einem s- und p-polarisiertem Laser in Abhängigkeit vom Einfallswinkel $\alpha$.}
  \label{tab:Messwerte}
  \begin{tblr}{colspec={c c c|| c c c|| c c c}}
      \toprule
      $\alpha\,[°]$ & $I_{\text{ref, s}}\,[\unit{\micro\ampere}]$ & $I_{\text{ref, p}}\,[\unit{\micro\ampere}]$ & $\alpha\,[°]$ & $I_{\text{ref, s}}\,[\unit{\micro\ampere}]$ & $I_{\text{ref, p}}\,[\unit{\micro\ampere}]$ & $\alpha\,[°]$ & $I_{\text{ref, s}}\,[\unit{\micro\ampere}]$ & $I_{\text{ref, p}}\,[\unit{\micro\ampere}]$ \\
      \midrule  
      6   &   6   &   14  &   38  &   24  &   20  &   70  &   110 &   3   \\
      8   &   8   &   14  &   40  &   31  &   20  &   71  &   120 &   1   \\
      10  &   7   &   15  &   42  &   28  &   20  &   72  &   120 &   2,2 \\
      12  &   10  &   15  &   44  &   39  &   20  &   73  &   130 &   1,4 \\
      14  &   6   &   14  &   46  &   38  &   20  &   74  &   140 &   0,9 \\
      16  &   11  &   11  &   48  &   47  &   20  &   75  &   140 &   0,5 \\
      18  &   10  &   16  &   50  &   46  &   20  &   76  &   130 &   0,57\\
      20  &   10  &   16  &   52  &   55  &   19  &   77  &   150 &   0,76\\
      22  &   12  &   16  &   54  &   64  &   17  &   78  &   140 &   1,5 \\
      24  &   15  &   17  &   56  &   70  &   16  &   79  &   160 &   2,8 \\
      26  &   12  &   17  &   58  &   70  &   16  &   80  &   150 &   4,6 \\
      28  &   17  &   18  &   60  &   80  &   14  &   82  &   170 &   10  \\
      30  &   14  &   18  &   62  &   78  &   12  &   84  &   160 &   23  \\
      32  &   19  &   19  &   64  &   90  &   10  &   86  &   190 &   43  \\
      34  &   18  &   19  &   66  &   90  &   8   &   87  &   190 &   60  \\
      36  &   26  &   19  &   68  &   110 &   5   &   &   & \\      
      \bottomrule
  \end{tblr}
\end{table}
Die berechneten Brechungsindizes sind in der Tabelle \ref{tab:Brechungsindex} aufgeführt. Ohne die Berücksichtigung von systematischen Fehlern, ergibt sich für die gemittelten Brechungsindizes
\begin{align*}
  \overline{n_{\text{s}}} &= 2,1\pm 1,1\quad \text{und}\\
  \overline{n_{\text{p}}} &= 4\pm 7\,.
\end{align*}
Insbesondere bei dem p-polarisierten Filter handelt es sich um einen großen Fehlerbereich. Daher werden aufgrund der systematischen Fehler für das s-polarisierte Licht die Werte $n_{\text{s}} > 4$ und bei dem p-polarisiertem 
Licht alle Werte $n_{\text{p}} > 6$ vernachlässigt. Die daraus gemittelten Brechungsindizes sind
\begin{align*}
  \overline{n_{\text{s}}} &= 2,0\pm 0,9\quad \text{und}\\
  \overline{n_{\text{p}}} &= 2,4\pm 0,7\,.
\end{align*}
Die Messdaten sind in der Abbildung \ref{fig:plot} dargestellt. Hierbei ist $\sqrt{\sfrac{I(\alpha)}{I_0}}$ gegen $\alpha$ aufgetragen. Zusätzlich sind in der Abbildung
die Theoriekurven abgebildet, welche durch $\overline{n_{\text{s}}}$, $\overline{n_{\text{p}}}$ sowie den Gleichungen (\ref{eqn:E_r_senkrecht}) und (\ref{eqn:E_r_parallel}) bestimmt werden.
Außerdem wird anhand der Tabelle beim Minimum des p-polarisitem Lasers ein Brewsterwinkel von $\alpha_{\text{B}} = 75\,°$ abgelesen. Dieser ist ebenfalls in der Abbildung \ref{fig:plot} eingezeichnet.
Anhand dieses Brewsterwinkels lässt sich über die Gleichung (\ref{eqn:Brewster}) der theoretische Brechungsindex $$n = 3,73$$ ermitteln.
\begin{figure}[H]
  % \centering
  \includegraphics[width=\textwidth]{plot.pdf}
  \caption{Graphische Darstellung der Messwerte mit der Theoriekurve und markiertem Brewsterwinkel.}
  \label{fig:plot}
\end{figure}                
\begin{table}[H]
    \centering
    \caption{Brechnete Brechungsindizes in Abhängigkeit des Winkels und der Intensität.}
    \label{tab:Brechungsindex}
    \begin{tblr}{colspec={c c c|| c c c|| c c c}}
        \toprule
        $\alpha\,[°]$ & $n_{\text{s}}$ & $n_{\text{p}}$ & $\alpha\,[°]$ & $n_{\text{s}}$ & $n_{\text{p}}$ & $\alpha\,[°]$ & $n_{\text{s}}$ & $n_{\text{p}}$ \\
        \midrule  
        6   &   1,00  &   1,43  & 38&   1,26  &  186   &   70  &   2,76 &   3,32\\
        8   &   1,01  &   1,43  & 40&   1,34  &  191   &   71  &   2,94 &   3,24\\
        10  &   1,01  &   1,46  & 42&   1,33  &  197   &   72  &   2,95 &   3,60\\
        12  &   1,02  &   1,46  & 44&   1,46  &  204   &   73  &   3,14 &   3,70\\
        14  &   1,02  &   1,45  & 46&   1,47  &  211   &   74  &   3,34 &   3,85\\
        16  &   1,03  &   1,40  & 48&   1,59  &  219   &   75  &   3,35 &   4,01\\
        18  &   1,04  &   1,51  & 50&   1,61  &  228   &   76  &   3,18 &   4,33\\
        20  &   1,05  &   1,52  & 52&   1,73  &  236   &   77  &   3,58 &   4,73\\
        22  &   1,06  &   1,54  & 54&   1,87  &  241   &   78  &   3,39 &   5,31\\
        24  &   1,09  &   1,58  & 56&   1,97  &  250   &   79  &   3,81 &   6,06\\
        26  &   1,08  &   1,60  & 58&   2,00  &  265   &   80  &   3,61 &   6,98\\
        28  &   1,12  &   1,64  & 60&   2,16  &  274   &   82  &   4,06 &   9,63\\
        30  &   1,12  &   1,67  & 62&   2,17  &  284   &   84  &   3,86 &   15,06\\
        32  &   1,17  &   1,72  & 64&   2,37  &  296   &   86  &   4,59 &   26,95\\
        34  &   1,18  &   1,76  & 66&   2,40  &  309   &   87  &   4,59 &   40,70\\
        36  &   1,25  &   1,80  & 68&   2,73  &  317   &   &   & \\      
        \bottomrule
    \end{tblr}
  \end{table}


